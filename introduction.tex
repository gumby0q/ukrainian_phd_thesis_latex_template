
\chapter*{Вступ}
\label{ch:introduction}

\begin{refsection}

\paragraph{Актуальність теми.}

Відомості щодо змін мікроструктури в приповерхневих шарах металевого прокату в результаті оброблення тиском, в деталях після проведення технологічних операцій зміцнення їх поверхонь або термохімічної модифікації, в електроенергетичному обладнанні внаслідок дії на нього механічних деформацій та перерозподілу концентрації пружних напружень тощо є важливою інформацією для впровадження у виробництво сучасних методів неруйнівного контролю якості виробів та матеріалів, забезпечення доброякісності виконання технологічних процесів моніторингу та діагностування критичних станів устаткування. Приповерхневі зміни мікроструктури об'єктів спостереження призводять до трансформації фізико-механічних поверхневих властивостей їх матеріалу. Отже, інформація щодо в'язкості, пластичності, твердості, теплоємності, міцності, а окрім того хімічного і фазового складу приповерхневого шару матеріалу, може бути отримана завдяки кореляційним зв'язкам фізико-механічних властивостей об'єктів дослідження з розподілами електричної провідності та магнітної проникності у приповерхневій зоні, тобто їх профілями. Це в свою чергу обумовило можливість використання для визначення профілів електрофізичних параметрів, котрі є високо структуро-чутливими, вихрострумового методу вимірювань. Слід зазначити, що значну частину об'єктів спостереження складають вироби, які характеризуються циліндричною геометричною формою. Тому є доцільним використання прохідних вихрострумових перетворювачів, вимірювання якими дозволяє в результаті розв'язку оберненої електродинамічної задачі відтворити радіальні профілі електричної провідності та магнітної проникності. 
Значний здобуток у розвиток теорії вихрострумового контролю внесений в тому числі й українськими вченими, зокрема Троїцьким~В.О., Маєвським~С.М., Гальченком~В.Я., Куцом~Ю.В., Сучковим~Г.М., Учаніним~В.М., Хандецьким~В.С..
Дослідженням питань, зв'язаних з вимірюваннями електрофізичних параметрів об'єктів дослідження вихрострумовим методом, присвячені роботи низки українських вчених, зокрема Назарчука~З.Т., Себка~В.П., Тетерка~А.Я., Горкунова~Б.М., Синявського~А.Т. тощо. Можна також відмітити неабиякий інтерес іноземних науковців до цієї тематики, серед яких варто відзначити Theodoulidis~T., Bowler~N., Ida~N., Lu~M., Tesfalem~H., Hampton~J., Huang~R., Burkhardt~J., Xu~J. та інших. Не зважаючи на достатньо глибоке опрацювання ними широкого кола питань щодо ідентифікації профілів електрофізичних параметрів об'єктів контролю (ОК), запропоновані методи виконання завдань таких задач не завжди є такими, що повною мірою реалізують існуючі на цей час вимоги. Зокрема дослідниками майже не приділялася увага розробленню засобів та методів визначення профілів електричної провідності (ЕП) та магнітної проникності (МП), здатних розв'язувати цю задачу водночас для обох розподілів в реальному масштабі часу з високою точністю по однократному результату реєстрації амплітуди та фази сигналу перетворювача вихрострумової вимірювальної системи. Тому варто сконцентруватися на подальшому розвитку відповідних досліджень щодо створення цілковито нових оригінальних підходів для досягнення цієї мети.
Отже, доцільність пропонованого дисертаційного дослідження визначається необхідністю ефективного вирішення зазначених вище питань, а прикладна задача, що розглядається у його рамках, є актуальною та представляє суттєвий науковий та практичний інтерес.

% За наявності у вступі можуть також вказуватися
\paragraph{Зв'язок роботи з науковими програмами, планами, темами.}

Дисертаційна робота виконана на кафедрі приладобудування, мехатроніки та комп'ютеризованих технологій Черкаського державного технологічного університету в період 2018 - 2024 р.р. у межах ініціативних науково-дослідницьких робіт за темами: “Обернені задачі вихрострумового контролю: моделі, алгоритми, методи оптимізації”, номер держреєстрації №0120U103875; “Розробка, дослідження експрес-методів вихрострумового вимірювання профілів електрофізичних параметрів об’єктів, що пройшли технологічні операції зміцнення поверхні”, номер держреєстрації №0122U200836, що відповідає напрямам досліджень, які започатковані в університеті. Здобувач як виконавець брав безпосередню участь в виконанні наведених досліджень.

\paragraph{Мета і завдання дослідження.}

Метою дисертаційної роботи є створення системи вихрострумового спільного вимірювання обох приповерхневих радіальних профілів електрофізичних характеристик циліндричних об’єктів контролю, реалізованої із застосуванням експрес-методу, який передбачає апріорне накопичення даних щодо них у нейромережевій сурогатній моделі та наступне її використання для підвищення точності вимірювань у динамічній таблиці другого рівня при швидкому пошуку розв'язку задачі за технологією Lookup tables.
Для досягнення мети дослідження необхідно виконання наступних завдань:
\begin{itemize}
  \item проведення аналізу предметної області, а саме існуючих методів та засобів вимірювання приповерхневих профілів електрофізичних характеристик ОК, виявлення їх недоліків та обґрунтування перспективних нових підходів до підвищення швидкодії, ефективності та точності їх визначення;
  \item обґрунтування та розроблення низки багатовимірних комп’ютерних \ldots
  \item \ldots
\end{itemize}


\textbf{Об’єкт дослідження} --- процеси вихрострумового вимірювального контролю струмопровідних об'єктів.
\par
\textbf{Предмет дослідження} --- вихрострумова система та експрес-метод вимірювання приповерхневих радіальних профілів електрофізичних характеристик циліндричних об'єктів прохідними перетворювачами.
\par
В процесі реалізації поставлених завдань були застосовані наступні \textbf{методи досліджень}: для опису процесів вихрострумового контролю циліндричних об’єктів - теорія електромагнітного поля, теорія інтегрального числення, теорія диференціальних рівнянь у частинних похідних, спеціальні функції математичної фізики, теорія матриць, чисельні методи, метод моделювання, теорія похибок; для створення однорідних ПЕ - теорія планування експериментів, методи математичної статистики, методи обчислювальної геометрії; для створення нейромережевих сурогатних моделей – теорія оптимізації, методи штучного інтелекту, методи машинного навчання, методи математичної статистики, теорія похибок. Для підтвердження обчислювальної ефективності запропонованих методів та визначення їх точності використано комп’ютерні експерименти.


\paragraph{Наукова новизна одержаних результатів.}

В процесі виконання поставлених завдань одержано наступні результати:
\begin{itemize}
  \item  вперше розроблено експрес-метод вимірювання радіальних приповерхневих профілів електричної провідності та магнітної проникності в об’єктах циліндричної форми, який відрізняється тим, що при застосуванні \ldots
  \item  \ldots
\end{itemize}



% За наявності у вступі можуть також вказуватися
\paragraph{Практичне значення отриманих результатів}

\begin{itemize}
  \item розроблено алгоритми та відповідні програмні засоби формування ефективних комп'ютерних однорідних багатовимірних квазі-планів експериментів із покращеною гомогенністю 2D-проєкцій та гарантовано низькими розбіжностями;
  \item \ldots
\end{itemize}

\paragraph{Використання результатів роботи.}

Результати проведених досліджень знайшли практичне впровадження в навчальних процесах кафедри приладобудування, мехатроніки та комп'ютеризованих технологій Черкаського державного технологічного університету; кафедри виробництва приладів...

\paragraph{Особистий внесок здобувача}

% Якщо у дисертації використано ідеї або розробки,
% що належать співавторам, разом з якими здобувачем опубліковано наукові праці,
% обов'язково зазначається
% конкретний особистий внесок здобувача в такі праці або розробки;
% здобувач має також додати посилання на дисертації співавторів,
% у яких було використано результати спільних робіт.


Усі наукові результати дисертаційної роботи автор отримав самостійно. У публікаціях, підготовлених в співавторстві, здобувачеві належать такі результати: у праці \cite{my_halchenko2019nonlinear} -  створення нейромережевої сурогатної моделі; у \cite{my_halchenko2020restoration} - створення програм обчислення сигналу від вихрострумових перетворювачів, проведення числових розрахунків; у \cite{my_halchenko2020construction} - візуалізація планів експерименту за допомогою діаграм Вороного, огляд методів створення квазі випадкових послідовностей; у \cite{my_halchenko2020surrogate} - створення сурогатної моделі прохідного перетворювача при контролі циліндричних об'єктів.


% Апробація матеріалів дисертації
\begin{approval}
% У вступі подається апробація матеріалів дисертації
% (зазначаються назви конференції, конгресу, симпозіуму, семінару, школи,
% місце та дата проведення).
Основні результати дисертаційної роботи доповідалися та підлягали обговоренню на таких Міжнародних та Всеукраїнських наукових конференціях: XV-та міжнародна конференція «Контроль i управління в складних системах», (м.~Вінниця, 2020); Міжнародний симпозіум «Проблеми електроенергетики,
електротехніки та електромеханіки» (м.~Харків, 2020); Міжнародна конференція «Дни на безразрушителния контрол» (м.~Созополь, Болгарія, 2020); Міжнародна науково-практична on-line конференція «Проблеми енергоефективності та автоматизації в промисловості та сільському господарстві» ( м.~Кропивницький, 2020); Міжнародних симпозіумах «Проблеми електроенергетики, електротехніки та електромеханіки» (м.~Харків, 2020 та 2023); XІ Міжнародна науково-технічна конференція «Датчики, прилади та системи» (м.~Черкаси, 2024).


% Основні результати дослідження доповідалися на наукових
% конференціях різного рівня та наукових семінарах. Це такі
% конференції:

Основні результати дослідження доповідалися на наукових
конференціях різного рівня. Це такі конференції:


\begin{itemize}
  % \item Український математичний конгрес, Київ, 21--23~серпня
  % 2001~р. \participation{секційна доповідь};

  \item V Міжнародній науково-практичній конференції
  "Інформаційні технології в освіті, науці і техніці"\ (ІТОНТ-2020)
  Черкаси, 21--22~травня 2020~р. \participation{онлайн доповідь};

  \item ХV Міжнародна конференція "Контроль і управління в складних системах"\ (КУСС-2020)
  Вінниця, 8--10~жовтня 2020~р. \participation{онлайн доповідь};


  % Команда \participation працює так,
  % що форму участі у вступі не буде показувати,
  % але якщо цей текст перенести в додаток, то буде.

  % \item Звітна конференція викладачів, аспірантів та докторантів університету,
  %   Київ, 1--2~лютого 2002~р. \participation{пленарна доповідь};

  % \item Конференція молодих вчених <<Сучасна алгебра і топологія>>,
  %   Одеса, 15--20~серпня 2003~р. \participation{стендова доповідь};

  \item \ldots
\end{itemize}
% Це такі семінари:
% \begin{itemize}
%   % \item семінар відділу теорії функцій Інституту математики НАН
%   % України (керівник: чл.-кор. НАН України О.",І.",Степанець);

%   % Говорити про форму участі в семінарі немає сенсу, на мій погляд.
%   % (Якщо розуміти семінар як такий науковий захід,
%   % де тільки один доповідач (або небагато), а всі інші "--- слухачі.
%   % Тоді не може бути ніякої пленарної, секційної чи стендової доповіді,
%   % як на конференції, де може бути кілька паралельних потоків.
%   % Можливо, в інших науках семінаром називають щось інше.)
%   % Але за потреби команду \participation тут теж можна використовувати.

%   \item \ldots
% \end{itemize}
\end{approval}

\paragraph{Публікації} Матеріали дисертаційного дослідження опубліковані у 27 наукових роботах, в тому числі 8 статтях, із яких 3 статті у закордонних періодичних наукових виданнях; 2 статті у виданнях, включених до переліку наукових фахових видань України; 4 статей у періодичних наукових виданнях, включених до наукометричної бази Web of Science; 1 стаття у періодичному науковому виданні, включеному до наукометричної бази Scopus; 1 стаття в періодичному закордонному фаховому виданні. Інші 19 публікацій - у матеріалах конференцій.

\paragraph{Структура та обсяг дисертації}

% Анонсується структура дисертації, зазначається її загальний обсяг.
Дисертаційна робота складається зі вступу, чотирьох розділів, загальних висновків, а також двох додатків. До кожного розділу наводиться список використаних джерел, що містить загалом 151 найменувань. Загальний обсяг дисертаційної роботи становить 160 сторінок, у тому числі 123 сторінки основного тексту, ілюстрованого 56 рисунками, який містить 30 таблиць.

% start 24
% end 183
% = 159 + 1


% \section{Список використаних джерел до розділу ~\ref{ch:introduction}}
\section*{Список використаних джерел до вступу}
\printbibliography[heading=none]
% \printbibliography[title={Список публікацій здобувача за~темою~дисертації}]
\end{refsection}