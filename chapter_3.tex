
\begin{refsection}

% Приклад назви розділу і мітки, на яку можна посилатися в тексті
\chapter{Алгоритмічне і програмне забезпечення для вимірювань профілів електрофізичних параметрів методом з апріорним накопиченням даних}
\label{ch:chapter_3}

\section{Програмне забезпечення для “точного” моделювання процесів вихрострумового контролю в об’єктах циліндричної форми}

Зазвичай для побудови точних електрофізичних моделей використовуються...

\section{Програмне забезпечення для створення комп’ютерних однорідних планів експериментів}\label{ch3:section2}


Для прикладу масштабування даних для реальних значень факторів взято базові значення з таблиці~\ref{tab:ch3:data_for_scaled_values} і ...
\begin{table}[htbp]
\small

\caption{Вихідні дані для створення масштабованих планів експерименту}
\label{tab:ch3:data_for_scaled_values}

\begin{center}
\begin{tabular}{|p{200pt}|c|c|c|c|}
    \hline
    & $\sigma_{51}$, См/м & $\mu_{r51}$ & $f$, Гц & $r$, мм \\ \hline
    Базове значення \newline (BaseValue) & 6990000 & 10 & 2500 & 10 \\ \hline
    Максимальне відносне відхилення & 15,00\% & 15,00\% & 20,00\% & 2,50\% \\ \hline
\end{tabular}
\end{center}
\end{table}


%----------------------------------

\begin{center}
    \small
    \begin{longtable}{|c|c|c|c|c|}
    \caption{Приклад масштабованого чотирьохфакторного плану експерименту}\label{tab:ch3:scaled_values_r4} \\
    \hline
    № & $\sigma_{51}$, См/м & $\mu_{r51}$ & $f$, Гц & $r$, мм \\ \hline
    \endfirsthead
    
    \multicolumn{5}{c}{\normalsize \bfseries Продовження таблиці \thetable} \\ [12pt] \hline
    № & $\sigma_{51}$, См/м & $\mu_{r51}$ & $f$, Гц & $r$, мм \\ \hline
    \endhead

    1 & 6,4852Е+06 & 8,7095Е+00 & 6,4852Е+06 & 8,7095Е+00 \\ \hline
    2 & 5,9705Е+06 & 1,0419Е+01 & 5,9705Е+06 & 1,0419Е+01 \\ \hline
    3 & 7,5557Е+06 & 9,1286Е+00 & 7,5557Е+06 & 9,1286Е+00 \\ \hline
    4 & 7,0410Е+06 & 1,0838Е+01 & 7,0410Е+06 & 1,0838Е+01 \\ \hline
    5 & 6,5262Е+06 & 9,5476Е+00 & 6,5262Е+06 & 9,5476Е+00 \\ \hline
    \ldots & \ldots & \ldots & \ldots & \ldots \\ \hline
    2498 & 6,3372Е+06 & 1,1383Е+01 & 6,3372Е+06 & 1,1383Е+01 \\ \hline
    2499 & 7,9224Е+06 & 1,0092Е+01 & 7,9224Е+06 & 1,0092Е+01 \\ \hline
    2500 & 7,4076Е+06 & 8,8019Е+00 & 7,4076Е+06 & 8,8019Е+00 \\ \hline
    2501 & 6,8929Е+06 & 1,0511Е+01 & 6,8929Е+06 & 1,0511Е+01 \\ \hline
    2502 & 6,3781Е+06 & 9,2210Е+00 & 6,3781Е+06 & 9,2210Е+00 \\ \hline
    \ldots & \ldots & \ldots & \ldots & \ldots \\ \hline
    4996 & 7,7743Е+06 & 9,7658Е+00 & 7,7743Е+06 & 9,7658Е+00 \\ \hline
    4997 & 7,2595Е+06 & 1,1475Е+01 & 7,2595Е+06 & 1,1475Е+01 \\ \hline
    4998 & 6,7448Е+06 & 1,0185Е+01 & 6,7448Е+06 & 1,0185Е+01 \\ \hline
    4999 & 6,2300Е+06 & 8,8944Е+00 & 6,2300Е+06 & 8,8944Е+00 \\ \hline
    5000 & 7,8153Е+06 & 1,0604Е+01 & 7,8153Е+06 & 1,0604Е+01 \\ \hline
\end{longtable} 
\end{center}


Наведений приклад показує...

\section{Висновки до розділу~\ref{ch:chapter_3}}

Алгоритмічне та програмне забезпечення для моделювання...

\section{Список використаних джерел до розділу ~\ref{ch:chapter_3}}
% \chapter*{Список використаних джерел до розділу ~\ref{ch:review}}
% \printbibliography[title={Reference List for Section 1}]
\printbibliography[heading=none]
\end{refsection}

