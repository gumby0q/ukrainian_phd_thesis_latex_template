%% Приклад розділу дисертації

\begin{refsection}

% Приклад назви розділу і мітки, на яку можна посилатися в тексті
\chapter{Огляд методів та засобів розв’язку задачі встановлення структурних особливостей матеріалу циліндричних об'єктів контролю}
\label{ch:chapter_1}
\section{Огляд методів визначення електрофізичних характеристик об’єктів вихрострумовим методом}
% Приклад назви підрозділу

Наразі є відомими чимала кількість варіантів щодо підходів розв’язку досліджуваної проблеми або суміжних задач вихрострумового контролю. Критичний аналіз відомих науково-технічної літератури показав особливості, доцільність і ефективність цих варіантів. Нижче наведений опис підходів та методів розв’язку обернених задач вихрострумового контролю із зазначенням їхніх переваг та недоліків.
В статті \cite{березюк2006розвязання} наведено приклад розв'язку оберненої задачі багатопараметрового контролю структурних змін матеріалу ОК змінно-частотним методом. Метод забезпечує контроль певної товщини шару матеріалу, що є корисним, наприклад, при контролі глибини термічної обробки матеріалу. Та хоч він і належить до багатопараметрових, що дозволяє контролювати інтегральний електромагнітний параметр $\eta = f (\mu, \sigma)$, але він не дозволяє окремо контролювати ЕП та МП. Багатопараметровий контроль кожного з параметрів можливий за допомогою окремої математичної моделі, що описує залежність вихідного сигналу від параметрів матеріалу. Підхід до розв'язку оберненої задачі є оптимізаційним на основі методу Флетчера–Пауела, що дозволяє оцінити відхилення виміряного та модельованого сигналу. Загалом метод є досить точним, але не використовує інформативність сигналу в повному обсязі (не включає амплітуду як інформативну складову), до того ж оптимізаційні підходи як правило не дозволяють проводити обробку сигналу в масштабах наближених до реального часу.

В публікації \cite{березюк2006розвязання} розв’язок оберненої задачі електродинаміки щодо реконструкції структури ОК за виміряними сигналами вихрострумового перетворювача (ВСП) рекомендується знаходити засобами лінійного програмування.

Лінійні припущення описані в роботі \cite{горкунов2011метод}, які використано при побудові математичної моделі рис.~\ref{fig:gorkunov.model}, а відповідно й запропонований  метод суперпозиції, не є строгими і значно спрощують реальні фізичні процеси. Крім того, при  проведенні вимірювальних операцій використовуються декілька частот, що ускладнює проведення процедури. 

\begin{figure}[htbp]
    \setlength{\unitlength}{1mm}
    \begin{center}
    \includegraphics[width=0.6\linewidth]{{ch1_images/gorkunov_2011_vsp_model.png}}
    % \includegraphics[scale=0.6]{{ch1_images/gorkunov_2011_vsp_model.pdf}}
    \caption{Модель ВСП з циліндричним ОК, поверхня якого являє собою двошарову котушку.}
    \label{fig:gorkunov.model}
    \end{center}
\end{figure}

У дослідженнях \cite{горкунов2018электромагнитный, gorkunov2018uncertainty} розглянуто електромагнітний перетворювач із просторово-періодичною структурою поля рис.~\ref{fig:gorkunov.roztashuvania}, що дозволяє проводити контроль та вимірювання параметрів ЕП та МП металевих виробів у формі протяжного феромагнітного циліндра. Автори пропонують використовувати специфічні гармоніки сигналу для визначення та виділення відхилень параметрів структури матеріалу ОК. Цей метод є досить вимогливим до якості сигналу самого ВСП та провздовжнє покриття циліндричного ОК робить його чутливість, а саме визначення локальних відхилень електрофізичних параметрів ОК, менш точною.

\begin{figure}[htbp]
    \setlength{\unitlength}{1mm}
    \begin{center}
    \includegraphics[width=0.6\linewidth]{{ch1_images/gorkunov_2018_uncertainity_fig1.png}}
    \caption{Розташування котушок збудження вздовж металевого циліндра, де 1 "--- провідник збудження, 2--5 "--- вимірювальні провідники, 6 "--- ОК}
    \label{fig:gorkunov.roztashuvania}
    \end{center}
\end{figure}

В наступній роботі \cite{горкунов2018экспериментальные} пропонується вимірювання параметрів ЕП та МП шляхом проведення вимірювання поздовжніми провідниками за спеціальною методологією зустрічного та паралельного включення струмів збудження та комбінацій позиціонування вимірювальних та збуджуючих обмоток рис.~\ref{fig:gorkunov.poperechnuy}. Метод є ефективним та досить перспективним в автоматизованих системах контролю, але вимога забезпечення комбінацій точного позиціонування котушок для системи вимірювання є недоліком, що призводить до ускладнення практичного застосування таких вимірювальних перетворювачів. Також для отримання інформативних результатів при спрощених конфігураціях запропонованих ВСП, а саме зменшення кількості вимірювальних та збуджуючих котушок, є необхідність проведення серії вимірювань з додатковим переміщеням ВСП навколо ОК рис.~\ref{fig:gorkunov.kombinacii}, що є небажаною технологічною операцією при впровадженні даного методу в виробництво.  Крім того, можливими є похибки просторового позиціювання вимірювальних обмоток, що призводить до додаткових похибок вимірювання.

\begin{figure}[htbp]
    \setlength{\unitlength}{1mm}
    \begin{center}
    \includegraphics[width=0.6\linewidth]{{ch1_images/gorkunov_2019_vpluv_fig2.png}}
    \caption{Поперечний переріз намагнічувальних систем зі струмами одного і того ж напряму та однаковими за величиною.}
    \label{fig:gorkunov.poperechnuy}
    \end{center}
\end{figure}

\begin{figure}[h]
    \setlength{\unitlength}{1mm}
    \begin{center}
    % \includegraphics[width=0.6\linewidth]{{ch1_images/gorkunov_2019_vpluv_fig4.png}}
    \includegraphics[scale=1.8]{{ch1_images/gorkunov_2019_vpluv_fig4.png}}
    \caption{Комбінації намагнічуючих систем перетворювачів з необхідною сумою переважаючих гармонік зондуючого поля.}
    \label{fig:gorkunov.kombinacii}
    \end{center}
\end{figure}

В публікаціях \cite{тетерко2011побудова, тетерко2010концепція, тетерко2014метод} вирішення проблеми вихрострумового контролю товщини оболонок виробів та захисних покриттів виробів, а також підвищення точності вимірювання контрольованих величин пропонується за допомогою введення автоматизованих систем з використанням зворотної функції перетворення та розрахунку її наближеного значення. Автори окреслюють основні принципи та вимоги, яким має відповідати апаратура контролю параметрів ОК, і показують конкретний приклад реалізації основних елементів такого вимірювального приладу. Так як задача контролю параметрів ОК є оберненою і нелінійною, модель зворотної функції перетворення представлено багатовимірним поліномом у базисі інформаційних параметрів.  Відповідно при використанні прямої функції перетворення відшукується розв’язок системи нелінійних рівнянь відносно невідомих параметрів. У випадку використання зворотної функції для визначення параметрів ОК застосовується знайдена нелінійна поліноміальна залежність від компонент вектора інформаційних параметрів перетворювача. Зазначений вище метод може бути застосований в масштабах реального часу та потребує відносно малих обчислювальних ресурсів на обрахунок результату. До недоліків методу слід віднести певні труднощі вибору структури полінома, що апріорі є невідомою, яка би забезпечила прийнятну похибку апроксимації гіперповерхні відгуку. Також відзначимо, що з ростом числа невідомих параметрів ОК (зазору, товщини ОК додатково), а відповідно розмірності гіперпростору, провести поліноміальну апроксимацію стає практично неможливим, а спрощена поліноміальна функція в свою чергу зменшує точність результату.


\section{Огляд методів створення сурогатних моделей із накопиченням апріорних даних}

Зазвичай розв’язання обернених задач в багатьох сферах науки та техніки потребують застосування оптимізаційних методів... 

\begin{figure}[htbp]
    \setlength{\unitlength}{1mm}
    \begin{center}
    \includegraphics[width=0.85\linewidth]{{ch1_images/diser__surogat_modeling.drawio.png}}
    \caption{Концепція сурогатного моделювання.}
    \label{fig:surogate_modeling}
    \end{center}
\end{figure}

\subsection{Евристичні моделі}

Розглядаючи клас евристичних сурогатних моделей...

\section{Аналіз методів створення комп’ютерних однорідних планів експериментів}

Як правило, для побудови адекватних та точних апроксимаційних моделей потрібні...

На рис.~\ref{fig:dist_compare} зображено однорідні вибірки різних двовимірних розподілень. Як видно при звичайному випадковому розподіленні точки схильні утворювати кластери і пробіли, а інші, квазі-випадкові послідовності з низькою розбіжністю, є повністю визначеними послідовностями, що розподіляються по простору так рівномірно наскільки це можливо.

% \begin{figure}[h]
\begin{figure}[!h]
    \setlength{\unitlength}{1mm}
    \begin{center}
    \begin{tabular}{cc} 
        {\includegraphics[scale=0.8]{{ch1_images/seq_types_rand_sequence.png}}}&
        {\includegraphics[scale=0.8]{{ch1_images/seq_types_halton_sequence.png}}}\\
        % 1.pdf figure&2.pdf figure\\
        а)&б)\\
        {\includegraphics[scale=0.8]{{ch1_images/seq_types_rd_sequence.png}}}&
        {\includegraphics[scale=0.8]{{ch1_images/seq_types_sobol_sequence.png}}}\\
        в)&г)\\
        % 2.pdf figure&1.pdf figure&\\
    \end{tabular}
    \caption{Порівняння двовимірних квазі-випадкових та простої випадкової послідовності: а) прості випадкові послідовності; б) послідовності Халтона; в) $R$ послідовності; г) послідовності Соболя.}
    \label{fig:dist_compare}
    \end{center}
\end{figure}

При малій розмірності (одновимірному і двовимірному просторі)  квазі-випадкові послідовності видаються менш однорідними ніж лінійне розбиття простору на однакові проміжки, але при збільшенні розмірності вони є найкращим вибором для формування вибірок.

\section{Огляд методів застосування штучних нейронних мереж для розв’язку обернених задач}

\subsection{Особливості розв’язку обернених задач з використанням нейромереж}

Застосування нейромереж в дослідженнях на сьогодні досить розповсюджена практика...
% \section*{Висновки до розділу~\ref{ch:chapter_1}}
\section{Висновки до розділу~\ref{ch:chapter_1}}

В даному розділі подано огляд застосування уже наявних та створення нових основних етапів засобів і методів  для ефективного розв’язку обернених задач...

% \nocite{Bar98fasp1, Pra98, Pie29} % Include the keys of the references you want in this section
% some text 1 \cite{Bar98fasp1}
% some text 2 \cite{Pie29}
% some text 3 \cite{Bar98fasp1, Pra98, Pie29}

% \section{Список літератури}
\section{Список використаних джерел до розділу ~\ref{ch:chapter_1}}
% \chapter*{Список використаних джерел до розділу ~\ref{ch:chapter_1}}
% \printbibliography[title={Reference List for Section 1}]
\printbibliography[heading=none]
\end{refsection}

