\begin{refsection}

% Приклад назви розділу і мітки, на яку можна посилатися в тексті
\chapter{Програмно-апаратний комплекс для вимірювання профілів електрофізичних параметрів циліндричних об’єктів контролю}
\label{ch:chapter_4}

% -------------------------------------------------------------------
\section{Апаратна частина комплексу}
% Приклад назви підрозділу

Однією з головних переваг підходу до розв’язку задачі...

\begin{figure}[htbp]
    \setlength{\unitlength}{1mm}
    \begin{center}
    \includegraphics[width=0.9\linewidth]{{ch4_images/hardware functional schematic 1.drawio.png}}
    \caption{Функціональна схема вихрострумового структуроскопу.}
    \label{fig:ch4:hardware_functional_schematic}
    \end{center}
\end{figure}


\begin{itemize}
    \item Генератор струму збудження, до якого входить генератор прямого цифрового синтезу (Direct digital synthesizer, DDS) та підсилювач сигналу сигнезатора. Він генерує синусоїдальний струм заданої частоти, який подається на обмотку збудження ВСП.

    \item ВСП з прохідним датчиком трансформаторного типу у вигляді двох котушок - котушки збудження та вимірювальної котушки.

    \item Модуль вимірювання параметрів...
\end{itemize}

% -------------------------------------------------------------------

% \begin{landscape}
    \begin{figure}[htbp]
        \setlength{\unitlength}{1mm}
        \begin{center}
        \includegraphics[width=\linewidth]{{ch4_images/sch_exitation_amp.png}}
        \caption{Електрична принципова схема підсилювача сигналу збудження.}
        \label{fig:ch4:sch_exitation_amp}
        \end{center}
    \end{figure}


    \begin{figure}[htbp]
        \setlength{\unitlength}{1mm}
        \begin{center}
        \includegraphics[width=0.9\linewidth]{{ch4_images/sch_signal_amp.png}}
        \caption{Електрична принципова схема підсилювача сигналу ВСП.}
        \label{fig:ch4:sch_signal_amp}
        \end{center}
    \end{figure}
% \end{landscape}

% landscape
\begin{landscape}
    \begin{figure}[htbp]
        \setlength{\unitlength}{1mm}
        \begin{center}
        \includegraphics[width=0.9\linewidth]{{ch4_images/sch_amplitude_and_phase_detectors.png}}
        \caption{Електричні принципові схеми фазового та амплітудного детектора сигналу ВСП з фільтрами низьких частот другого порядку.}
        \label{fig:ch4:sch_amplitude_and_phase_detectors}
        \end{center}
    \end{figure}

\end{landscape}

% -------------------------------------------------------------------

% -------------------------------------------------------------------
\section{Висновки до розділу~\ref{ch:chapter_4}}

Даний розділ присвячений опису...

\section{Список використаних джерел до розділу ~\ref{ch:chapter_4}}
% \chapter*{Список використаних джерел до розділу ~\ref{ch:review}}
% \printbibliography[title={Reference List for Section 1}]
\printbibliography[heading=none]
\end{refsection}

