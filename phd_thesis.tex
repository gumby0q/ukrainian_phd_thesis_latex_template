
%% Приклад головного файла дисертації (для mon2017dev.cls)

% Це просто файл-приклад, який дисертантки і дисертанти можуть використати
% як основу для своєї дисертації.
% Але в цьому базовому вигляді він може не підходити для всіх дисертацій.
% Треба наповнити його конкретним змістом,
% а також зробити певні налаштування згідно зі своїми потребами:
% вибрати потрібні опції класу,
% підключити корисні додаткові пакунки,
% налаштувати параметри сторінки (зокрема розміри берегів),
% означити специфічні команди, які використовуються в дисертації, тощо.

\documentclass[
  type=phd,% тип дисертації
    %%%%%% Ступені згідно з Порядком присудження наукових ступенів (Постанова КМ)
    % c    кандидат <галузь науки> (типово)
    % d    доктор <галузь науки>
    %%%%%% Ступені згідно із Законом України «Про вищу освіту»
    % phd  доктор філософії
    % artd доктор мистецтва
    % scd  доктор наук
  % instnameorder=asc,% порядок розміщення на титульному аркуші
                      % назви установи і назви органу, до сфери управління якого належить установа
    % desc низхідний (типово)
    % asc  висхідний
    % ascd висхідний порядок з відмінюванням
  % guide=mon2017dev.naiau,% модуль для підтримки специфічних вимог
    % karazin Харківський національний університет імені В. Н. Каразіна
    %         -- галузь науки для типу дисертації c і d
    %         -- дрібні зміни в оформленні
    % knu     Київський національний університет імені Тараса Шевченка
    %         -- повтор однакових установ двічі
    % naiau   Національна академія внутрішніх справ
    %         -- лапки, тире точно, як у зразку з наказу МОН
    %         -- дрібні зміни в оформленні
    % --------------------------------------------------------------------------
    % kivi    у стилі Киви
    %         -- дисертація за здоров'я наукового рівня
    %            dissertation to the health of the science level
]{mon2017dev}[2021/07/21]

% Налагодження кодування шрифта, кодування вхідного файла
% та вибір необхідних мов
% \usepackage[T2A]{fontenc}
% \usepackage[T2A]{fontenc}

% Для шрифтів T1 треба встановити cm-super
% \usepackage[T1, T2A]{fontenc}
\usepackage[T1]{fontenc}
% \usepackage[T2A]{fontenc}

% \setsansfont{YourSansCyrillicFont}
\usepackage{fontspec}
\setmainfont{Times New Roman}
% \setmainfont{Noto Serif}
% \setmainfont{Times New Roman}[
%     SmallCapsFont = {Times New Roman}
% ]
% \setmainfont{Times New Roman}[
%     SmallCapsFont = {Noto Serif}
% ]
\usepackage{graphicx}

% \usepackage[cp1251]{inputenc}
\usepackage[utf8]{inputenc}
% TODO: Перевірити, якщо головна мова документа -- english.
% Ще не всі службові слова перекладено (тільки ті, що використовуються в анотації).
% Не рекомендовано використовувати для документів,
% у яких головна мова -- не ukrainian (а english чи інша).
\usepackage[russian,english,ukrainian]{babel}

% bibliography things -------------------------------------------------------------->>>>
% \usepackage[backend=biber, style=numeric-comp, sorting=none]{biblatex} % numeric-comp - vancouver style -> only that works somehow
% \usepackage[backend=biber, style=ieee, sorting=none]{biblatex}
\usepackage[backend=biber, style=ieee, sorting=none, autolang=other]{biblatex} % numeric-comp - vancouver style

% bibliography formating in other than default language:
% in  \usepackage[backend=biber, style=ieee, sorting=none, autolang=other]{biblatex} add !!!autolang=other!!!
% ensure that this language is in the document
% in .bib add the field to needed item
% langid       = {english}

% sorting https://tex.stackexchange.com/a/51439
% Define the formatting for citation numbers
\DeclareFieldFormat{labelnumberwidth}{{#1. }}
% \usepackage[backend=biber, style=apa]{biblatex}
% \usepackage[backend=biber, style=ieee]{biblatex}
% \usepackage[backend=biber, style=mla]{biblatex}
\addbibresource{abstract_references.bib}
\addbibresource{chapter_1_references.bib}
\addbibresource{chapter_2_references.bib}
\addbibresource{chapter_3_references.bib}
\addbibresource{chapter_4_references.bib}
% bibliography things --------------------------------------------------------------<<<<

% Підключення необхідних пакетів. Наприклад,
% Пакети AMS для підтримки математики, теорем, спеціальних шрифтів
\usepackage[intlimits]{amsmath}
\allowdisplaybreaks
\usepackage{amsthm}
\usepackage{amssymb}
% Налагодження нумерованих списків
\usepackage{enumerate}
% Гіпертекстові документи
%\usepackage{hyperref}
% або лише спеціальне форматування URL
%\usepackage{url}
% У списку літератури зворотні вказівки на посилання
%\usepackage{backref}
% Сортування посилань
%\usepackage[noadjust]{cite}
% Останні два пакети несумісні між собою. Крім того, конфліктують з цим класом!
% Таблиці зі стовпчиками, що розтягуються
\usepackage{tabularx}
% Включення факсимільних підписів і взагалі робота з ілюстраціями, кольором тощо
% \usepackage{graphicx}

% ---------------------------------------------------------------------------------->>>>
% Для вимоги оприлюднює … на своєму офіційному веб-сайті електронну копію дисертації у форматі PDF/A
\usepackage{pdfx}
% ----------------------------------------------------------------------------------<<<<
\usepackage{multirow}
\usepackage{longtable}

\usepackage{pdflscape} % for landscape
% ----------------------------------------------------------------------------------<<<<


% Налагодження параметрів сторінки (зокрема берегів).
% Наприклад, за допомогою пакета geometry
% УВАГА!
% Параметри сторінки не зафіксовані жорстко вимогами до оформлення дисертацій
% і можуть за потреби дещо змінюватися.
% Це просто приклад.
% Варто встановити свої параметри залежно від особливостей тексту дисертації
% чи технічних особливостей майстерні, яка робитиме оправу.
\usepackage{geometry}
\geometry{hmargin={30mm,15mm},lines=29,vcentering}

% Означення теорем (теоремоподібних структур)
% Класичний варіант: для кожної теореми свій лічильник,
% тобто теорема 1.1, лема 1.1, теорема 1.2
\theoremstyle{plain}
\newtheorem{theorem}{Теорема}[chapter]
\newtheorem{lemma}{Лема}[chapter]
\newtheorem{corollary}{Наслідок}[chapter]
\theoremstyle{definition}
\newtheorem{definition}{Означення}[chapter]
\newtheorem{example}{Приклад}[chapter]
\theoremstyle{remark}
\newtheorem{remark}{Зауваження}[chapter]
% Цікавий варіант: всі теореми нумеруються одним лічильником,
% тобто теорема 1.1, лема 1.2, теорема 1.3
%\theoremstyle{plain}
%\newtheorem{theorem}{Теорема}[chapter]
%\newtheorem{lemma}[theorem]{Лема}
%\newtheorem{corollary}[theorem]{Наслідок}
%\theoremstyle{definition}
%\newtheorem{definition}[theorem]{Означення}
%\newtheorem{example}[theorem]{Приклад}
%\theoremstyle{remark}
%\newtheorem{remark}[theorem]{Зауваження}

% Локальні означення
% \newcommand{\N}{\mathbb{N}}
% \newcommand{\Z}{\mathbb{Z}}
% \newcommand{\Q}{\mathbb{Q}}
% \newcommand{\R}{\mathbb{R}}
% \newcommand{\set}[1]{\left\{#1\right\}}
% \newcommand{\abs}[1]{\left\lvert#1\right\rvert}
% \newcommand{\norm}[2][]{\left\lVert#2\right\rVert_{#1}}
% Це потрібно для скороченого запису ряду Остроградського
% \newcommand{\Osign}[1]{\mathrm{O}^{#1}}

% Якщо потрібно працювати лише з деякими розділами
%\includeonly{xampl-ch1,xampl-bib}

% Інформація про використані пакети тощо.
% Може знадобитися для відлагодження класу документа
%\listfiles


% Text wrapping fixes 
% https://en.wikibooks.org/wiki/LaTeX/Text_Formatting#Hyphenation
% \pretolerance=150
% https://texfaq.org/FAQ-overfull.html
\setlength{\emergencystretch}{3em}


\begin{document}

% Назва дисертації
% \title(uk){Система вихрострумового вимірювання приповерхневих радіальних профілів електрофізичних характеристик циліндричних об'єк\discretionary{}{}{}тів}
\title(uk){Система вихрострумового вимірювання приповерхневих радіальних профілів електрофізичних характеристик циліндричних об'єктів}
% \title(uk){Система вихрострумового вимірювання приповерхневих \hyphenation{ра-ді-аль-них про-фі-лів еле-ктро-фі-зи-чних ха-ра-кте-ри-стик ци-лін-дри-чних об’єк-тів}}

\title(en){System of eddy current measurement of subsurface radial profiles of electrophysical characteristics of cylindrical objects}

% Прізвище, ім'я, по батькові здобувача
\author(uk){Сторчак Анатолій Вячеславович}
\author(en){Storchak Anatolii Viacheslavovych}

% Факсимільний підпис автора у файлі sorokina-sig.pdf, .eps, .jpeg тощо
% (зсув по x, зсув по y)
% [параметри команди \includegraphics]
% \facsimilesig{author}(-60,-12)[width=70pt]{sorokina-sig}

% Прізвище, ім'я, по батькові наукового керівника/консультанта
\supervisor(uk){Гальченко Володимир Якович}
% Науковий ступінь, вчене звання наукового керівника/консультанта
               {доктор технічних наук, професор}
% Установа, де працює науковий керівник/консультант, і посада
               {Черкаський державний технологічний університет,
               професор кафедри приладобудування, мехатроніки та комп'ютеризованих технологій}
% \supervisor(en){Redko Valeriy Hryhorovych}
%                {candidate of pedagogical sciences, associate professor}
%                {}
% \supervisor(uk){Альфа Бета Гамма}
%                {доктор фізико-математичних наук, професор}
%                {}
% \supervisor(en){Alpha Beta Gamma}
%                {doctor of physical and mathematical sciences, professor}
%                {}

% Спеціальність
% Якщо код спеціальності присутній у CSV-файлі, то клас читає інформацію з нього.
% Потрібний CSV-файл вибирається залежно від типу дисертації та основної мови документа.
% За потреби можна підключити інший CSV-файл з шифрами і назвами спеціальностей
% або у факультативному аргументі команди задати всі дані.
% Для типу дисертації c або d
% можна використовувати обидві команди \specialitysci і \specialityedu одночасно,
% щоб отримати наче «змішаний» ступінь:
% тоді на титульному аркуші й в анотації
% буде написано шифри й назви спеціальностей за обома переліками,
% а для типу дисертації c ще й ступінь у вигляді
% «кандидата ... наук (доктора філософії)»

% Варіант для дисертацій на здобуття наукового ступеня кандидата чи доктора ... наук
% (згідно з Переліком наукових спеціальностей)
% Цю команду треба використовувати для типу дисертації c або d.
% \specialitysci(uk)[
%   % specialityname=Теорія і методика професійної освіти,% спеціальність
%   % degreefield=педагогічні,                            % галузь науки, за якою присуджується науковий ступінь
%   % specialityfile=<filename>.csv
% ]{13.00.04}                                             % шифр спеціальності
% \specialitysci(en)[
%   % specialityname=Theory and methodology of professional education,
%   % degreefield=pedagogical,
%   % specialityfile=<filename>.csv
% ]{13.00.04}

% Варіант для дисертацій на здобуття наукового ступеня доктора філософії, доктора мистецтва чи доктора наук
% (згідно з Переліком галузей знань і спеціальностей, за якими здійснюється підготовка здобувачів вищої освіти
% Цю команду треба використовувати для типу дисертації phd, artd або scd,
% але також можна для типу дисертації c або d разом з командою \specialitysci.
\specialityedu(uk)[
  specialityname=Метрологія та інформаційно-вимірювальна техніка,% найменування спеціальності
  % fieldcode=01,                   % шифр галузі
  % fieldname=Metrology,    % галузь знань
  % specialityfile=<filename>.csv
]{152}                              % код спеціальності
\specialityedu(en)[
  specialityname=Metrology and information-measuring technique,
  % fieldcode=01,
  % fieldname=Metrology,
  % specialityfile=<filename>.csv
]{152}

% Звичайно, доступна команда \speciality, як у попередніх версіях.
% Клас означує її як одну з команд \specialitysci чи \specialityedu
% залежно від типу дисертації.
% Тому можна використовувати її у такій формі, наприклад:
% \speciality(uk)[degreefield={фізико"=математичні}]{01.01.01}
% \speciality(en){01.01.01}

% Індекс за УДК
\udc{620.179.147:519.853.6}

% Установа, де виконана робота, і місто
\institution(uk)[
  altname=Черкаський державний технологічний університет% альтернативна назва установи (пишеться в анотації в описі дисертації)
% ]{Черкаський державний технологічний університет}
]{}
 {Черкаси}
\institution(en)[
  altname=Cherkasy State Technological University
% ]{Cherkasy State Technological University}
]{}
 {Cherkasy}

% Команду \council переписано в стилі команди \institution:
% ключ institution задає «стандартну» назву установи, у спеціалізованій вченій раді якої проводиться захист дисертації
% (з вказуванням назви органу, до сфери управління якого належить заклад, установа),
% а ключ altname — «альтернативну» (тобто скорочену, для анотації).
% Якщо факультативний аргумент відсутній, то клас вважає,
% що захист проводиться в тій самій установі, де здійснювалася підготовка здобувача,
% а отже, немає потреби писати назву цієї установи двічі на титульному і в анотації.
% Але за потреби можна вручну повторити тут назву установи, і буде повтор на титульному і в анотації.
\council(uk)[
  institution={Черкаський державний технологічний університет,
    Міністерство освіти і науки України},
  altname=Черкаський державний технологічний університет,
  address={18006, м.~Черкаси, б-р~Шевченка, 460},
  % town=Київ
]{Д~26.053.01}
\council(en)[
  institution={Cherkasy State Technological University,
    Ministry of Education and Science of Ukraine},
  altname=Cherkasy State Technological University,
  % town=Kyiv
]{D~26.053.01}

% Рік, коли написана дисертація
\date{2024}

% \secret{Таємно}

% Тут буде титульний аркуш
\maketitle

% Анотація
%% mon2017dev-abs.tex  Приклад анотації (для mon2017dev.tex)

\begin{abstract}[
  language=ukrainian,% мова анотації
  % chapter=Реферат, % заголовок розділу або false, щоб не робити заголовок (типово Анотація/Abstract)
  % header=false     % автоматична генерація опису дисертації (типово true)
]

В дисертаційній роботі було розв’язано задачу щодо вимірювання приповерхневих радіальних профілів електрофізичних характеристик циліндричних об’єктів та описано процес створення такої системи вимірювання на основі реалізованої сурогатної моделі процесу вихрострумового контролю...


  \keywords{%
    вихрострумове вимірювання, 
    моніторинг,
    циліндричний об’єкт контролю, 
    радіальні профілі магнітної проникності і електричної провідності, 
    сурогатна модель, 
    нейромережа, 
    однорідний комп’ютерний план експерименту,
    ...
  }
\end{abstract}

\begin{abstract}[
  language=english,
  % chapter=Summary,
  % header=false
]
  

The dissertation solves the problem of measuring the subsurface radial profiles of the electrophysical properties of cylindrical objects and describes the process of creating such a measurement system based on the implemented surrogate model of the eddy current testing process...

  \keywords{%
    eddy current measurement, 
    monitoring,
    cylindrical test object, 
    radial profiles of magnetic permeability and electrical conductivity, 
    surrogate model, 
    neural network,
    uniform computer design of experiment,
    ...
  }
\end{abstract}


\begin{refsection}
\nocite{
  my_halchenko2019nonlinear,
  my_halchenko2020restoration,
  my_halchenko2020construction,
  my_halchenko2020surrogate,
  my_halchenko2022measurement,
  my_trembovetska2020linear,
  my_halchenko2020methods,
  my_trembovetska2019accuracy,
  my_storchak2017eddy,
  my_storchak2018neural,
  my_storchak2018mathematical,
  my_galchenko2019surrogate,
}


% {\Large \textbf{Custom Heading Without Section}}\par
\vspace{1em}
\centering
{ \textbf{Список публікацій здобувача за~темою~дисертації}}\par
% Add some vertical space after the custom heading
\vspace{1em}

\printbibliography[heading=none]
% \printbibliography[title={Список публікацій здобувача за~темою~дисертації}]
\end{refsection}


% \nocite{Bar98fasp1,Bar98fasp2,PrB01umc}

% \begin{bibset}% [a]
%   {Список публікацій здобувача за~темою~дисертації}
%   % {Список публікацій здобувача}
%   \bibliographystyle{gost2008}
%   %
%   % Якщо не треба нумерація з крапкою, можна закоментувати наступні три рядки.
%   \makeatletter
%   \renewcommand\@biblabel[1]{#1.}
%   \makeatother
%   \bibliography{xampl-mybib}
% \end{bibset}


% Зміст
\tableofcontents

% Розділи дисертації в окремих файлах
%%
%% This is file `xampl-ch1.tex',
%% generated with the docstrip utility.
%%
%% The original source files were:
%%
%% vakthesis.dtx  (with options: `xampl-ch1')
%% 
%% IMPORTANT NOTICE:
%% 
%% For the copyright see the source file.
%% 
%% Any modified versions of this file must be renamed
%% with new filenames distinct from xampl-ch1.tex.
%% 
%% For distribution of the original source see the terms
%% for copying and modification in the file vakthesis.dtx.
%% 
%% This generated file may be distributed as long as the
%% original source files, as listed above, are part of the
%% same distribution. (The sources need not necessarily be
%% in the same archive or directory.)
%% xampl-ch1.tex  Приклад розділу дисертації


% Приклад назви розділу і мітки, на яку можна посилатися в тексті
\chapter*{Перелік умовних позначень}
\label{ch:acronyms}

ОК "--- об'єкт контролю;

ЕП "--- електрична провідність;

МП "--- магнітна проникність;

ПЕ "--- план експерименту;

ЕРС "--- електрорушійна сила;

LUT "--- lookup table;

ВСП "--- вихрострумовий перетворювач;

APDL "--- ANSYS parametric design language;

ШНМ, НМ "--- штучна нейронна мережа;

MARs "--- multivariate adaptive regression splines;

NURBs "--- non-uniform rational B-splines;

ANN "--- artificial neural networks;

RBF "--- radial basis function;

MLP "--- multilayer perceptron;

SVM "--- support vector machine;

GMDH "--- group method of data handling;

TPs "--- thin plate splines;

DNN, DANN "--- deep artificial neural networks;

FEM "--- finite element method;

RL "--- reinforcement learning;

GAN "--- generative adversarial network;

AE "--- autoencoder;

VAE "--- variational autoencoder;

INN "--- invertible neural network;

DL "--- deep learning;

CVNN "--- complex-valued neural network;

SCVNN "--- splittable complex-valued neural network;

RVNN "--- real-valued neural network;

MAE "--- mean absolute error;

MSE "--- mean square error;

RMSE "--- root mean square error;

MAPE "--- mean absolute percentage error;

DDS "--- direct digital synthesizer.


% Приклад назви підрозділу



% Вступ

\chapter*{Вступ}
\label{ch:introduction}

\begin{refsection}

\paragraph{Актуальність теми.}

Відомості щодо змін мікроструктури в приповерхневих шарах металевого прокату в результаті оброблення тиском, в деталях після проведення технологічних операцій зміцнення їх поверхонь або термохімічної модифікації, в електроенергетичному обладнанні внаслідок дії на нього механічних деформацій та перерозподілу концентрації пружних напружень тощо є важливою інформацією для впровадження у виробництво сучасних методів неруйнівного контролю якості виробів та матеріалів, забезпечення доброякісності виконання технологічних процесів моніторингу та діагностування критичних станів устаткування. Приповерхневі зміни мікроструктури об'єктів спостереження призводять до трансформації фізико-механічних поверхневих властивостей їх матеріалу. Отже, інформація щодо в'язкості, пластичності, твердості, теплоємності, міцності, а окрім того хімічного і фазового складу приповерхневого шару матеріалу, може бути отримана завдяки кореляційним зв'язкам фізико-механічних властивостей об'єктів дослідження з розподілами електричної провідності та магнітної проникності у приповерхневій зоні, тобто їх профілями. Це в свою чергу обумовило можливість використання для визначення профілів електрофізичних параметрів, котрі є високо структуро-чутливими, вихрострумового методу вимірювань. Слід зазначити, що значну частину об'єктів спостереження складають вироби, які характеризуються циліндричною геометричною формою. Тому є доцільним використання прохідних вихрострумових перетворювачів, вимірювання якими дозволяє в результаті розв'язку оберненої електродинамічної задачі відтворити радіальні профілі електричної провідності та магнітної проникності. 
Значний здобуток у розвиток теорії вихрострумового контролю внесений в тому числі й українськими вченими, зокрема Троїцьким~В.О., Маєвським~С.М., Гальченком~В.Я., Куцом~Ю.В., Сучковим~Г.М., Учаніним~В.М., Хандецьким~В.С..
Дослідженням питань, зв'язаних з вимірюваннями електрофізичних параметрів об'єктів дослідження вихрострумовим методом, присвячені роботи низки українських вчених, зокрема Назарчука~З.Т., Себка~В.П., Тетерка~А.Я., Горкунова~Б.М., Синявського~А.Т. тощо. Можна також відмітити неабиякий інтерес іноземних науковців до цієї тематики, серед яких варто відзначити Theodoulidis~T., Bowler~N., Ida~N., Lu~M., Tesfalem~H., Hampton~J., Huang~R., Burkhardt~J., Xu~J. та інших. Не зважаючи на достатньо глибоке опрацювання ними широкого кола питань щодо ідентифікації профілів електрофізичних параметрів об'єктів контролю (ОК), запропоновані методи виконання завдань таких задач не завжди є такими, що повною мірою реалізують існуючі на цей час вимоги. Зокрема дослідниками майже не приділялася увага розробленню засобів та методів визначення профілів електричної провідності (ЕП) та магнітної проникності (МП), здатних розв'язувати цю задачу водночас для обох розподілів в реальному масштабі часу з високою точністю по однократному результату реєстрації амплітуди та фази сигналу перетворювача вихрострумової вимірювальної системи. Тому варто сконцентруватися на подальшому розвитку відповідних досліджень щодо створення цілковито нових оригінальних підходів для досягнення цієї мети.
Отже, доцільність пропонованого дисертаційного дослідження визначається необхідністю ефективного вирішення зазначених вище питань, а прикладна задача, що розглядається у його рамках, є актуальною та представляє суттєвий науковий та практичний інтерес.

% За наявності у вступі можуть також вказуватися
\paragraph{Зв'язок роботи з науковими програмами, планами, темами.}

Дисертаційна робота виконана на кафедрі приладобудування, мехатроніки та комп'ютеризованих технологій Черкаського державного технологічного університету в період 2018 - 2024 р.р. у межах ініціативних науково-дослідницьких робіт за темами: “Обернені задачі вихрострумового контролю: моделі, алгоритми, методи оптимізації”, номер держреєстрації №0120U103875; “Розробка, дослідження експрес-методів вихрострумового вимірювання профілів електрофізичних параметрів об’єктів, що пройшли технологічні операції зміцнення поверхні”, номер держреєстрації №0122U200836, що відповідає напрямам досліджень, які започатковані в університеті. Здобувач як виконавець брав безпосередню участь в виконанні наведених досліджень.

\paragraph{Мета і завдання дослідження.}

Метою дисертаційної роботи є створення системи вихрострумового спільного вимірювання обох приповерхневих радіальних профілів електрофізичних характеристик циліндричних об’єктів контролю, реалізованої із застосуванням експрес-методу, який передбачає апріорне накопичення даних щодо них у нейромережевій сурогатній моделі та наступне її використання для підвищення точності вимірювань у динамічній таблиці другого рівня при швидкому пошуку розв'язку задачі за технологією Lookup tables.
Для досягнення мети дослідження необхідно виконання наступних завдань:
\begin{itemize}
  \item проведення аналізу предметної області, а саме існуючих методів та засобів вимірювання приповерхневих профілів електрофізичних характеристик ОК, виявлення їх недоліків та обґрунтування перспективних нових підходів до підвищення швидкодії, ефективності та точності їх визначення;
  \item обґрунтування та розроблення низки багатовимірних комп’ютерних \ldots
  \item \ldots
\end{itemize}


\textbf{Об’єкт дослідження} --- процеси вихрострумового вимірювального контролю струмопровідних об'єктів.
\par
\textbf{Предмет дослідження} --- вихрострумова система та експрес-метод вимірювання приповерхневих радіальних профілів електрофізичних характеристик циліндричних об'єктів прохідними перетворювачами.
\par
В процесі реалізації поставлених завдань були застосовані наступні \textbf{методи досліджень}: для опису процесів вихрострумового контролю циліндричних об’єктів - теорія електромагнітного поля, теорія інтегрального числення, теорія диференціальних рівнянь у частинних похідних, спеціальні функції математичної фізики, теорія матриць, чисельні методи, метод моделювання, теорія похибок; для створення однорідних ПЕ - теорія планування експериментів, методи математичної статистики, методи обчислювальної геометрії; для створення нейромережевих сурогатних моделей – теорія оптимізації, методи штучного інтелекту, методи машинного навчання, методи математичної статистики, теорія похибок. Для підтвердження обчислювальної ефективності запропонованих методів та визначення їх точності використано комп’ютерні експерименти.


\paragraph{Наукова новизна одержаних результатів.}

В процесі виконання поставлених завдань одержано наступні результати:
\begin{itemize}
  \item  вперше розроблено експрес-метод вимірювання радіальних приповерхневих профілів електричної провідності та магнітної проникності в об’єктах циліндричної форми, який відрізняється тим, що при застосуванні \ldots
  \item  \ldots
\end{itemize}



% За наявності у вступі можуть також вказуватися
\paragraph{Практичне значення отриманих результатів}

\begin{itemize}
  \item розроблено алгоритми та відповідні програмні засоби формування ефективних комп'ютерних однорідних багатовимірних квазі-планів експериментів із покращеною гомогенністю 2D-проєкцій та гарантовано низькими розбіжностями;
  \item \ldots
\end{itemize}

\paragraph{Використання результатів роботи.}

Результати проведених досліджень знайшли практичне впровадження в навчальних процесах кафедри приладобудування, мехатроніки та комп'ютеризованих технологій Черкаського державного технологічного університету; кафедри виробництва приладів...

\paragraph{Особистий внесок здобувача}

% Якщо у дисертації використано ідеї або розробки,
% що належать співавторам, разом з якими здобувачем опубліковано наукові праці,
% обов'язково зазначається
% конкретний особистий внесок здобувача в такі праці або розробки;
% здобувач має також додати посилання на дисертації співавторів,
% у яких було використано результати спільних робіт.


Усі наукові результати дисертаційної роботи автор отримав самостійно. У публікаціях, підготовлених в співавторстві, здобувачеві належать такі результати: у праці \cite{my_halchenko2019nonlinear} -  створення нейромережевої сурогатної моделі; у \cite{my_halchenko2020restoration} - створення програм обчислення сигналу від вихрострумових перетворювачів, проведення числових розрахунків; у \cite{my_halchenko2020construction} - візуалізація планів експерименту за допомогою діаграм Вороного, огляд методів створення квазі випадкових послідовностей; у \cite{my_halchenko2020surrogate} - створення сурогатної моделі прохідного перетворювача при контролі циліндричних об'єктів.


% Апробація матеріалів дисертації
\begin{approval}
% У вступі подається апробація матеріалів дисертації
% (зазначаються назви конференції, конгресу, симпозіуму, семінару, школи,
% місце та дата проведення).
Основні результати дисертаційної роботи доповідалися та підлягали обговоренню на таких Міжнародних та Всеукраїнських наукових конференціях: XV-та міжнародна конференція «Контроль i управління в складних системах», (м.~Вінниця, 2020); Міжнародний симпозіум «Проблеми електроенергетики,
електротехніки та електромеханіки» (м.~Харків, 2020); Міжнародна конференція «Дни на безразрушителния контрол» (м.~Созополь, Болгарія, 2020); Міжнародна науково-практична on-line конференція «Проблеми енергоефективності та автоматизації в промисловості та сільському господарстві» ( м.~Кропивницький, 2020); Міжнародних симпозіумах «Проблеми електроенергетики, електротехніки та електромеханіки» (м.~Харків, 2020 та 2023); XІ Міжнародна науково-технічна конференція «Датчики, прилади та системи» (м.~Черкаси, 2024).


% Основні результати дослідження доповідалися на наукових
% конференціях різного рівня та наукових семінарах. Це такі
% конференції:

Основні результати дослідження доповідалися на наукових
конференціях різного рівня. Це такі конференції:


\begin{itemize}
  % \item Український математичний конгрес, Київ, 21--23~серпня
  % 2001~р. \participation{секційна доповідь};

  \item V Міжнародній науково-практичній конференції
  "Інформаційні технології в освіті, науці і техніці"\ (ІТОНТ-2020)
  Черкаси, 21--22~травня 2020~р. \participation{онлайн доповідь};

  \item ХV Міжнародна конференція "Контроль і управління в складних системах"\ (КУСС-2020)
  Вінниця, 8--10~жовтня 2020~р. \participation{онлайн доповідь};


  % Команда \participation працює так,
  % що форму участі у вступі не буде показувати,
  % але якщо цей текст перенести в додаток, то буде.

  % \item Звітна конференція викладачів, аспірантів та докторантів університету,
  %   Київ, 1--2~лютого 2002~р. \participation{пленарна доповідь};

  % \item Конференція молодих вчених <<Сучасна алгебра і топологія>>,
  %   Одеса, 15--20~серпня 2003~р. \participation{стендова доповідь};

  \item \ldots
\end{itemize}
% Це такі семінари:
% \begin{itemize}
%   % \item семінар відділу теорії функцій Інституту математики НАН
%   % України (керівник: чл.-кор. НАН України О.",І.",Степанець);

%   % Говорити про форму участі в семінарі немає сенсу, на мій погляд.
%   % (Якщо розуміти семінар як такий науковий захід,
%   % де тільки один доповідач (або небагато), а всі інші "--- слухачі.
%   % Тоді не може бути ніякої пленарної, секційної чи стендової доповіді,
%   % як на конференції, де може бути кілька паралельних потоків.
%   % Можливо, в інших науках семінаром називають щось інше.)
%   % Але за потреби команду \participation тут теж можна використовувати.

%   \item \ldots
% \end{itemize}
\end{approval}

\paragraph{Публікації} Матеріали дисертаційного дослідження опубліковані у 27 наукових роботах, в тому числі 8 статтях, із яких 3 статті у закордонних періодичних наукових виданнях; 2 статті у виданнях, включених до переліку наукових фахових видань України; 4 статей у періодичних наукових виданнях, включених до наукометричної бази Web of Science; 1 стаття у періодичному науковому виданні, включеному до наукометричної бази Scopus; 1 стаття в періодичному закордонному фаховому виданні. Інші 19 публікацій - у матеріалах конференцій.

\paragraph{Структура та обсяг дисертації}

% Анонсується структура дисертації, зазначається її загальний обсяг.
Дисертаційна робота складається зі вступу, чотирьох розділів, загальних висновків, а також двох додатків. До кожного розділу наводиться список використаних джерел, що містить загалом 151 найменувань. Загальний обсяг дисертаційної роботи становить 160 сторінок, у тому числі 123 сторінки основного тексту, ілюстрованого 56 рисунками, який містить 30 таблиць.

% start 24
% end 183
% = 159 + 1


% \section{Список використаних джерел до розділу ~\ref{ch:introduction}}
\section*{Список використаних джерел до вступу}
\printbibliography[heading=none]
% \printbibliography[title={Список публікацій здобувача за~темою~дисертації}]
\end{refsection}% Вступ
%% Приклад розділу дисертації

\begin{refsection}

% Приклад назви розділу і мітки, на яку можна посилатися в тексті
\chapter{Огляд методів та засобів розв’язку задачі встановлення структурних особливостей матеріалу циліндричних об'єктів контролю}
\label{ch:chapter_1}
\section{Огляд методів визначення електрофізичних характеристик об’єктів вихрострумовим методом}
% Приклад назви підрозділу

Наразі є відомими чимала кількість варіантів щодо підходів розв’язку досліджуваної проблеми або суміжних задач вихрострумового контролю. Критичний аналіз відомих науково-технічної літератури показав особливості, доцільність і ефективність цих варіантів. Нижче наведений опис підходів та методів розв’язку обернених задач вихрострумового контролю із зазначенням їхніх переваг та недоліків.
В статті \cite{березюк2006розвязання} наведено приклад розв'язку оберненої задачі багатопараметрового контролю структурних змін матеріалу ОК змінно-частотним методом. Метод забезпечує контроль певної товщини шару матеріалу, що є корисним, наприклад, при контролі глибини термічної обробки матеріалу. Та хоч він і належить до багатопараметрових, що дозволяє контролювати інтегральний електромагнітний параметр $\eta = f (\mu, \sigma)$, але він не дозволяє окремо контролювати ЕП та МП. Багатопараметровий контроль кожного з параметрів можливий за допомогою окремої математичної моделі, що описує залежність вихідного сигналу від параметрів матеріалу. Підхід до розв'язку оберненої задачі є оптимізаційним на основі методу Флетчера–Пауела, що дозволяє оцінити відхилення виміряного та модельованого сигналу. Загалом метод є досить точним, але не використовує інформативність сигналу в повному обсязі (не включає амплітуду як інформативну складову), до того ж оптимізаційні підходи як правило не дозволяють проводити обробку сигналу в масштабах наближених до реального часу.

В публікації \cite{березюк2006розвязання} розв’язок оберненої задачі електродинаміки щодо реконструкції структури ОК за виміряними сигналами вихрострумового перетворювача (ВСП) рекомендується знаходити засобами лінійного програмування.

Лінійні припущення описані в роботі \cite{горкунов2011метод}, які використано при побудові математичної моделі рис.~\ref{fig:gorkunov.model}, а відповідно й запропонований  метод суперпозиції, не є строгими і значно спрощують реальні фізичні процеси. Крім того, при  проведенні вимірювальних операцій використовуються декілька частот, що ускладнює проведення процедури. 

\begin{figure}[htbp]
    \setlength{\unitlength}{1mm}
    \begin{center}
    \includegraphics[width=0.6\linewidth]{{ch1_images/gorkunov_2011_vsp_model.png}}
    % \includegraphics[scale=0.6]{{ch1_images/gorkunov_2011_vsp_model.pdf}}
    \caption{Модель ВСП з циліндричним ОК, поверхня якого являє собою двошарову котушку.}
    \label{fig:gorkunov.model}
    \end{center}
\end{figure}

У дослідженнях \cite{горкунов2018электромагнитный, gorkunov2018uncertainty} розглянуто електромагнітний перетворювач із просторово-періодичною структурою поля рис.~\ref{fig:gorkunov.roztashuvania}, що дозволяє проводити контроль та вимірювання параметрів ЕП та МП металевих виробів у формі протяжного феромагнітного циліндра. Автори пропонують використовувати специфічні гармоніки сигналу для визначення та виділення відхилень параметрів структури матеріалу ОК. Цей метод є досить вимогливим до якості сигналу самого ВСП та провздовжнє покриття циліндричного ОК робить його чутливість, а саме визначення локальних відхилень електрофізичних параметрів ОК, менш точною.

\begin{figure}[htbp]
    \setlength{\unitlength}{1mm}
    \begin{center}
    \includegraphics[width=0.6\linewidth]{{ch1_images/gorkunov_2018_uncertainity_fig1.png}}
    \caption{Розташування котушок збудження вздовж металевого циліндра, де 1 "--- провідник збудження, 2--5 "--- вимірювальні провідники, 6 "--- ОК}
    \label{fig:gorkunov.roztashuvania}
    \end{center}
\end{figure}

В наступній роботі \cite{горкунов2018экспериментальные} пропонується вимірювання параметрів ЕП та МП шляхом проведення вимірювання поздовжніми провідниками за спеціальною методологією зустрічного та паралельного включення струмів збудження та комбінацій позиціонування вимірювальних та збуджуючих обмоток рис.~\ref{fig:gorkunov.poperechnuy}. Метод є ефективним та досить перспективним в автоматизованих системах контролю, але вимога забезпечення комбінацій точного позиціонування котушок для системи вимірювання є недоліком, що призводить до ускладнення практичного застосування таких вимірювальних перетворювачів. Також для отримання інформативних результатів при спрощених конфігураціях запропонованих ВСП, а саме зменшення кількості вимірювальних та збуджуючих котушок, є необхідність проведення серії вимірювань з додатковим переміщеням ВСП навколо ОК рис.~\ref{fig:gorkunov.kombinacii}, що є небажаною технологічною операцією при впровадженні даного методу в виробництво.  Крім того, можливими є похибки просторового позиціювання вимірювальних обмоток, що призводить до додаткових похибок вимірювання.

\begin{figure}[htbp]
    \setlength{\unitlength}{1mm}
    \begin{center}
    \includegraphics[width=0.6\linewidth]{{ch1_images/gorkunov_2019_vpluv_fig2.png}}
    \caption{Поперечний переріз намагнічувальних систем зі струмами одного і того ж напряму та однаковими за величиною.}
    \label{fig:gorkunov.poperechnuy}
    \end{center}
\end{figure}

\begin{figure}[h]
    \setlength{\unitlength}{1mm}
    \begin{center}
    % \includegraphics[width=0.6\linewidth]{{ch1_images/gorkunov_2019_vpluv_fig4.png}}
    \includegraphics[scale=1.8]{{ch1_images/gorkunov_2019_vpluv_fig4.png}}
    \caption{Комбінації намагнічуючих систем перетворювачів з необхідною сумою переважаючих гармонік зондуючого поля.}
    \label{fig:gorkunov.kombinacii}
    \end{center}
\end{figure}

В публікаціях \cite{тетерко2011побудова, тетерко2010концепція, тетерко2014метод} вирішення проблеми вихрострумового контролю товщини оболонок виробів та захисних покриттів виробів, а також підвищення точності вимірювання контрольованих величин пропонується за допомогою введення автоматизованих систем з використанням зворотної функції перетворення та розрахунку її наближеного значення. Автори окреслюють основні принципи та вимоги, яким має відповідати апаратура контролю параметрів ОК, і показують конкретний приклад реалізації основних елементів такого вимірювального приладу. Так як задача контролю параметрів ОК є оберненою і нелінійною, модель зворотної функції перетворення представлено багатовимірним поліномом у базисі інформаційних параметрів.  Відповідно при використанні прямої функції перетворення відшукується розв’язок системи нелінійних рівнянь відносно невідомих параметрів. У випадку використання зворотної функції для визначення параметрів ОК застосовується знайдена нелінійна поліноміальна залежність від компонент вектора інформаційних параметрів перетворювача. Зазначений вище метод може бути застосований в масштабах реального часу та потребує відносно малих обчислювальних ресурсів на обрахунок результату. До недоліків методу слід віднести певні труднощі вибору структури полінома, що апріорі є невідомою, яка би забезпечила прийнятну похибку апроксимації гіперповерхні відгуку. Також відзначимо, що з ростом числа невідомих параметрів ОК (зазору, товщини ОК додатково), а відповідно розмірності гіперпростору, провести поліноміальну апроксимацію стає практично неможливим, а спрощена поліноміальна функція в свою чергу зменшує точність результату.


\section{Огляд методів створення сурогатних моделей із накопиченням апріорних даних}

Зазвичай розв’язання обернених задач в багатьох сферах науки та техніки потребують застосування оптимізаційних методів... 

\begin{figure}[htbp]
    \setlength{\unitlength}{1mm}
    \begin{center}
    \includegraphics[width=0.85\linewidth]{{ch1_images/diser__surogat_modeling.drawio.png}}
    \caption{Концепція сурогатного моделювання.}
    \label{fig:surogate_modeling}
    \end{center}
\end{figure}

\subsection{Евристичні моделі}

Розглядаючи клас евристичних сурогатних моделей...

\section{Аналіз методів створення комп’ютерних однорідних планів експериментів}

Як правило, для побудови адекватних та точних апроксимаційних моделей потрібні...

На рис.~\ref{fig:dist_compare} зображено однорідні вибірки різних двовимірних розподілень. Як видно при звичайному випадковому розподіленні точки схильні утворювати кластери і пробіли, а інші, квазі-випадкові послідовності з низькою розбіжністю, є повністю визначеними послідовностями, що розподіляються по простору так рівномірно наскільки це можливо.

% \begin{figure}[h]
\begin{figure}[!h]
    \setlength{\unitlength}{1mm}
    \begin{center}
    \begin{tabular}{cc} 
        {\includegraphics[scale=0.8]{{ch1_images/seq_types_rand_sequence.png}}}&
        {\includegraphics[scale=0.8]{{ch1_images/seq_types_halton_sequence.png}}}\\
        % 1.pdf figure&2.pdf figure\\
        а)&б)\\
        {\includegraphics[scale=0.8]{{ch1_images/seq_types_rd_sequence.png}}}&
        {\includegraphics[scale=0.8]{{ch1_images/seq_types_sobol_sequence.png}}}\\
        в)&г)\\
        % 2.pdf figure&1.pdf figure&\\
    \end{tabular}
    \caption{Порівняння двовимірних квазі-випадкових та простої випадкової послідовності: а) прості випадкові послідовності; б) послідовності Халтона; в) $R$ послідовності; г) послідовності Соболя.}
    \label{fig:dist_compare}
    \end{center}
\end{figure}

При малій розмірності (одновимірному і двовимірному просторі)  квазі-випадкові послідовності видаються менш однорідними ніж лінійне розбиття простору на однакові проміжки, але при збільшенні розмірності вони є найкращим вибором для формування вибірок.

\section{Огляд методів застосування штучних нейронних мереж для розв’язку обернених задач}

\subsection{Особливості розв’язку обернених задач з використанням нейромереж}

Застосування нейромереж в дослідженнях на сьогодні досить розповсюджена практика...
% \section*{Висновки до розділу~\ref{ch:chapter_1}}
\section{Висновки до розділу~\ref{ch:chapter_1}}

В даному розділі подано огляд застосування уже наявних та створення нових основних етапів засобів і методів  для ефективного розв’язку обернених задач...

% \nocite{Bar98fasp1, Pra98, Pie29} % Include the keys of the references you want in this section
% some text 1 \cite{Bar98fasp1}
% some text 2 \cite{Pie29}
% some text 3 \cite{Bar98fasp1, Pra98, Pie29}

% \section{Список літератури}
\section{Список використаних джерел до розділу ~\ref{ch:chapter_1}}
% \chapter*{Список використаних джерел до розділу ~\ref{ch:chapter_1}}
% \printbibliography[title={Reference List for Section 1}]
\printbibliography[heading=none]
\end{refsection}

%        Розділ 1

\begin{refsection}

% Приклад назви розділу і мітки, на яку можна посилатися в тексті
\chapter{Метод вимірювання профілів електрофізичних параметрів матеріалу циліндричних об’єктів з апріорним накопиченням даних}
\label{ch:chapter_2}
\section{Концептуальна постановка задачі}
% Приклад назви підрозділу

Визначення приповерхневих радіальних профілів електрофізичних характеристик...


\section{“Точна” електродинамічна модель процесу вимірювання вихрострумовим перетворювачем електрофізичних параметрів циліндричних об’єктів контролю}
\label{ch:chapter_2:sec:precize_analytic_model}

Математична модель складалася на основі загальних законів теорії електромагнітного поля Максвелла та описує процес контролю прохідним круговим ВСП циліндричного співвісного ОК. Є відомими аналітичні моделі процесу вихрострумового контролю циліндричних ОК \cite{koliskina2013analytical, kolyshkin1994analytical}, але внаслідок універсальності має сенс використовувати модель Dodd-Deeds \cite{nestor1979analysis}. Для спрощення представлення профілів розподілу електрофізичних параметрів у її контексті, пропонується використовувати кусково-постійну апроксимацію, коли ОК вважається умовно багатошаровим та електрофізичні параметри в кожному n-му шарі із його $(K–1)$ шарів приймаються сталими: $\sigma_n=\sigma(r)$, $\mu_n=\mu(r)$, де $n=1,2,\ldots,(K-1)$ рис.~\ref{fig:ch2:model_with_rod_layers}.

\begin{figure}[h]
    \setlength{\unitlength}{1mm}
    \begin{center}
    \includegraphics[width=0.75\linewidth]{{ch2_images/model_with_rod_layers.jpg}}
    \caption{Геометрична модель прохідного ВСП з циліндричним ОК: $r_{d1}$, $r_{d2}$ "--- внутрішній та зовнішній радіуси котушки збудження відповідно; $r_n$ "--- зовнішній радіус $n$-го шару; $l_{d1}$, $l_{d2}$ "--- відстані до граней котушки збудження, $r_s$ "--- радіус вимірювального витка, $l_s$ "--- відстань до вимірювального витка.}
    \label{fig:ch2:model_with_rod_layers}
    \end{center}
\end{figure} 


Для коректного опису математичної моделі введено поняття регіону. Кожен регіон у циліндричній системі координат може бути описаний системою таких нерівностей:

\begin{equation*}
    \begin{aligned}
    {R_1} &= \left\{ \right.0 \le r \le r_1,\, 0 \le {\varphi } \le 2 \pi,\, -\infty < z < \infty \left.\right\} \\ 
    & \quad \quad \quad \quad \quad \quad \quad \quad \quad \vdots \\
    {R_i} &= \left\{ \right. {r_{i - 1}} \le r \le {r_i},\;0 \le {\varphi } \le 2{{\pi }},\; - \infty  < z < \infty \left.\right\}\\
    & \quad \quad \quad \quad \quad \quad \quad \quad \quad \vdots \\
    {R_N} &= \left\{ \right. {r_{N - 1}} < r,\;0 \le {{\varphi }} \le 2{{\pi }},\; - \infty  < z < \infty \left.\right\},
\end{aligned}
\end{equation*}

де $i = 2,3,...,(N - 1)$, а $N$ "--- загальна кількість регіонів, ${r_{N - 1}} = {r_d}$. 

Математична модель складалася при прийнятті таких припущень \cite{nestor1979analysis}: середовища є лінійними, ізотропними та однорідними; струм збудження I є синусоїдальним, що змінюється з кутовою частотою $\omega$. Котушка збудження на початковому етапі розглядається як нескінченно тонкий виток з радіусом $r_d$. Також приймається, що осі ВСП та циліндричного ОК співпадають. 
Густина струму збудження та векторний потенціал у циліндричній системі координат за таких умов мають тільки азимутальну складову:

\begin{equation}
    \vec A\left( {r,\;{{\varphi }},\;z} \right) = A\left( {r,\;z} \right){\vec e_{{\varphi }}}, \, \vec J\left( {r,\;{{\varphi }},\;z} \right) = J\left( {r,\;z} \right){\vec e_{{\varphi }}}.
\end{equation}

Диференціальні рівняння для векторного потенціалу в регіонах із номерами $(N-1)$ та $N$ можна записати у вигляді:

\[{R_{N - 1}} \cup {R_N}:\;\quad \frac{{{\partial ^2}A}}{{\partial {r^2}}} + \frac{1}{r}\frac{{\partial A}}{{\partial r}} - \frac{A}{{{r^2}}} + \frac{{{\partial ^2}A}}{{\partial {z^2}}} = 0\]

а в регіонах ${R_1},{R_2},...,{R_{N - 2}}$ відповідно:

\begin{equation*}
\frac{{{\partial ^2}{A^{(i)}}}}{{\partial {r^2}}} + \frac{1}{r}\frac{{\partial {A^{(i)}}}}{{\partial r}} - \frac{{{A^{(i)}}}}{{{r^2}}} + \frac{{{\partial ^2}{A^{(i)}}}}{{\partial {z^2}}} = j{{\omega }}{{{\mu }}_i}{{{\sigma }}_i}{A^{(i)}},
\end{equation*}

де $i = 1,2,...,(N - 2)$, $\mu_i$ "--- абсолютна магнітна проникність, $j = \sqrt { - 1}$.

Враховуючи, що для векторного потенціалу з фізичних міркувань виконуються такі умови: а) $A$ є скінченним при $r=0$, б) $A \rightarrow 0$ при $r \rightarrow \infty$, та беручи до уваги граничні умови:

\begin{equation*}
\begin{aligned}
{\left. {{A^{(i)}}\left( {r,\;z} \right)} \right|_{r = {r_i}}} &= {\left. {{A^{(i + 1)}}\left( {r,\;z} \right)} \right|_{r = {r_i}}},\\
{\left. {\frac{1}{{{{{\mu }}_i}}}\frac{{\partial {A^{(i)}}}}{{\partial r}}\left( {r,z} \right)} \right|_{r = {r_i}}} &= {\left. {\frac{1}{{{{{\mu }}_{i + 1}}}}\frac{{\partial {A^{(i + 1)}}}}{{\partial r}}\left( {r,z} \right)} \right|_{r = {r_i}}},
\end{aligned}
\end{equation*}

де $i = 1,2,...,(N - 2)$

\begin{equation*}
\begin{aligned}
    {\left. {{A^{(N - 1)}}\left( {r,\;z} \right)} \right|_{r = {r_{N - 1}}}} &= {\left. {{A^{(N)}}\left( {r,\;z} \right)} \right|_{r = {r_{N - 1}}}}, \\
    {\left. {\frac{1}{{{{{\mu }}_{N - 1}}}}\frac{{\partial {A^{(N - 1)}}}}{{\partial r}}\left( {r,z} \right)} \right|_{r = {r_{N - 1}}}} &= {\left. {\frac{1}{{{{{\mu }}_N}}}\frac{{\partial {A^{(N)}}}}{{\partial r}} \left( {r,z} \right)} \right|_{r = {r_{N - 1}}}} + I{{\delta }}\left( {r - {r_d}} \right){{\delta }}\left( {z - {z_d}} \right),
\end{aligned}
\end{equation*}

де $\delta$ "--- дельта-функція Дірака, отримано рівняння для векторного потенціалу в будь-якому регіоні всередині витка зі струмом, що має такий вигляд:

\begin{equation}
\begin{aligned}
    A\left( {r,z,{r_d},{z_d}} \right) &= \frac{{I{{{\mu }}_0}{r_d}}}{{{\pi }}}\mathop \int_{0}^{\infty} \frac{{Q1\,Q2}}{{{U_{22}}{V_{11}} - {U_{12}}{V_{21}}}}\cos \left( {{{\alpha }}(z - {z_d})} \right) \, d{{\alpha }},\\
    Q1 &= {V_{11}}{I_1}\left( {{{{\alpha }}_n}r} \right) + {V_{21}}{K_1}\left( {{{{\alpha }}_n}r} \right),\\
    Q2 &= {U_{12}}{I_1}\left( {{{\alpha }}{r_d}} \right) + {U_{22}}{K_1}\left( {{{\alpha }}{r_d}} \right),
\end{aligned}    
\end{equation}

де ${{{\mu }}_0} = 4{{\pi }} \cdot {10^{ - 7}}$ Гн/м "--- магнітна стала;

\begin{equation*}
\begin{aligned}
    {V_{11}}\left( {n + 1,n} \right) &= \left( {{K_0}\left( {{{{\alpha }}_{n + 1}}{r_n}} \right){I_1}\left( {{{{\alpha }}_n}{r_n}} \right) + \frac{{{{{\beta }}_n}}}{{{{{\beta }}_{n + 1}}}}{I_0}\left( {{{{\alpha }}_n}{r_n}} \right){K_1}\left( {{{{\alpha }}_{n + 1}}{r_n}} \right)} \right){{{\alpha }}_{n + 1}}{r_n};\\
    {U_{12}}\left( {n + 1,n} \right) &= \bigg( {K_0}\left( {{{{\alpha }}_{n + 1}}{r_n}} \right){K_1}\left( {{{{\alpha }}_n}{r_n}} \right) \\ & \left.- \frac{{{{{\beta }}_n}}}{{{{{\beta }}_{n + 1}}}}{K_0}\left( {{{{\alpha }}_n}{r_n}} \right){K_1}\left( {{{{\alpha }}_{n + 1}}{r_n}} \right) \right){{{\alpha }}_{n + 1}}{r_n};\\
    {V_{21}}\left( {n + 1,n} \right) &= \left( {{I_0}\left( {{{{\alpha }}_{n + 1}}{r_n}} \right){I_1}\left( {{{{\alpha }}_n}{r_n}} \right) - \frac{{{{{\beta }}_n}}}{{{{{\beta }}_{n + 1}}}}{I_0}\left( {{{{\alpha }}_n}{r_n}} \right){I_1}\left( {{{{\alpha }}_{n + 1}}{r_n}} \right)} \right){{{\alpha }}_{n + 1}}{r_n};\\
    {U_{22}}\left( {n + 1,n} \right) &= \left( {{I_0}\left( {{{{\alpha }}_{n + 1}}{r_n}} \right){K_1}\left( {{{{\alpha }}_n}{r_n}} \right) + \frac{{{{{\beta }}_n}}}{{{{{\beta }}_{n + 1}}}}{K_0}\left( {{{{\alpha }}_n}{r_n}} \right){I_1}\left( {{{{\alpha }}_{n + 1}}{r_n}} \right)} \right){{{\alpha }}_{n + 1}}{r_n};\\
    {{{\beta }}_n} &= \left( {\frac{{{{{\mu }}_0}}}{{{{{\mu }}_n}}}} \right){{{\alpha }}_n};
\end{aligned}    
\end{equation*}

${I_0},{I_1}$ "--- модифіковані функції Бесселя першого роду нульового та першого порядків від комплексного аргументу;

${K_0},{K_1}$ "--- модифіковані функції Бесселя другого роду нульового та першого порядків від комплексного аргументу;

${{{\alpha }}_n} = \;\sqrt {{{{\alpha }}^2} - j{{{\mu }}_n}{{{\sigma }}_n}{{\;}}}$, $n = 1,2,...,K$.

Векторний потенціал в області в середині котушки збудження, яка має прямокутний поперечний переріз та однорідний розподіл густини струму збудження, можна записати у вигляді:

\begin{equation}
    A\left( {r,z} \right) = I{N_d}\int_{{l_{d1}}}^{{l_{d2}}} {\int_{{r_{d1}}}^{{r_{d2}}} {A\left( {r,z,{r_d},{z_d}} \right) \,d{r_d} \,d{z_d}} }
\end{equation}

де ${N_d} = \frac{W}{{\left( {{r_{d2}} - {r_{d1}}} \right)\left( {{l_{d2}} - {l_{d1}}} \right)}}$; $W$ "--- кількість витків котушки збудження.

Зміною порядку інтегрування та в результаті його виконання отримано вираз для векторного потенціалу:

\begin{equation} \label{eq:vecpot_1}
\begin{aligned}
    A\left( {r,z} \right) &= \frac{{I{N_d}{{{\mu }}_0}{r_d}}}{{{\pi }}} \int_0^\infty  \frac{{Q1\,\,Q2}}{{{{{\alpha }}^3}\left( {{U_{22}}{V_{11}} - {U_{12}}{V_{21}}} \right)}}Q3 \, d{{\alpha }},\\
    Q1 &= \sin \left( {{{\alpha }}(z - {l_{d1}})} \right) - \sin \left( {{{\alpha }}(z - {l_{d2}})} \right),\\
    Q2 &= {V_{11}}{I_1}\left( {{{{\alpha }}_n}r} \right) + {V_{21}}{K_1}\left( {{{{\alpha }}_n}r} \right),\\
    Q3 &= {U_{12}}I\left( {{r_{d2}},{r_{d1}}} \right) + {U_{22}}K\left( {{r_{d2}},{r_{d1}}} \right),
\end{aligned}
\end{equation}

де $I\left( {{r_{d2}},{r_{d1}}} \right) = \int_{\alpha {r_{d1}}}^{\alpha {r_{d2}}} t{I_1}\left( {{{\alpha }}t} \right) \,dt$;

$K\left( {{r_{d2}},{r_{d1}}} \right) = \int_{\alpha {r_{d1}}}^{\alpha {r_{d2}}} t{K_1}\left( {{{\alpha }}t} \right) \,dt$.

Наведена в круговому вимірювальному витку напруга з урахуванням \eqref{eq:vecpot_1} обчислюється відповідно до співвідношення:

\begin{equation} \label{eq:voltage_1}
    E = j\omega\oint_{l_{s-coil}} \vec{A} \,dl=j\omega2\pi r_{s}A(r_s,z_s).
\end{equation}

Таким чином, задаючи обране радіальне розподілення параметрів матеріалу ОК, модель дозволяє отримати необхідний відгук у вигляді напруги в комплексній формі, а це в свою чергу робить її перспективною до використання в даному дисертаційному дослідженні.

\section{Створення сурогатної моделі процесу контролю з апріорним накопиченням інформації}

В оглядовому розділі було зроблено...

В \cite{uzal1992theory} було запропоновано базові функції апроксимації розподілення параметрів і наведено приклад детальної апроксимації приповерхневої зони товщиною 1мм 50-ма умовними шарами. Для дисертаційного дослідження було обрано чотири типи таких функцій, які подано в табл.~\ref{tab:profiles_approximation} \cite{uzal1992theory,koliskina2013analytical}. На практиці профілі електрофізичних характеристик, які вважатимемо “нормою”, тобто взірцем, що отриманий внаслідок коректної технологічної поверхневої обробки ОК одним із відомих способів, можуть бути визначені експериментально.

\begin{table}[h]
% \begin{table}[htbp]
% \begin{table}[h]
    \caption{Апроксимаційні функції профілів розподілу електрофізичних параметрів}
    \label{tab:profiles_approximation}
    % some tips https://stackoverflow.com/questions/3068555/how-to-insert-manual-line-breaks-inside-latex-tables
    % \begin{tabularx}{\textwidth}{p{3cm}|X|c|c|c|}
    \begin{tabularx}{\textwidth}{|X|c|c|c|}
    \hline
    Вид \newline апроксимації & Апроксимаційна модель & Графічне зображення \\
    \hline
    Гаусіан &  
        $\begin{array}{@{}c@{}}
        \sigma (r) = {\sigma _1} + ({\sigma _2} - {\sigma _1}){e^{\frac{ - {r^2}}{g^2}}} \\
        \mu (r) = {{{\mu }}_1} + ({{{\mu }}_2} - {{{\mu }}_1}){e^{\frac{ - {r^2}}{g^2}}}
        \end{array}$
    &   \begin{minipage}{0.3\textwidth}
        \includegraphics[width=4.5cm, height=3.2cm]{ch2_images/distribution_gausian.png}
        \end{minipage} \\
    \hline
    гіперболічний тангенс & 
        $\begin{array}{@{}c@{}}
            \sigma(r) = {{{\sigma }}_1} + \frac{{{{{\sigma }}_2} - {{{\sigma }}_1}}}{2}\left( {1 + \tanh \frac{{r + c}}{{2a}}} \right)\\
            \mu (r) = {{{\mu }}_1} + \frac{{{{{\mu }}_2} - {{{\mu }}_1}}}{2}\left( {1 + \tanh \frac{{r + c}}{{2a}}} \right)
        \end{array}$
    &   \begin{minipage}{0.3\textwidth}
        \includegraphics[width=4.5cm, height=3.2cm]{ch2_images/distribution_tanhiperbolic.png}
        \end{minipage} \\
    \hline
    експоненціальна функція &
        $\begin{array}{@{}c@{}}
            {{\sigma }}(r) = {{{\sigma }}_1} + ({{{\sigma }}_2} - {{{\sigma }}_1}){e^{\frac{r}{b}}} \\
            {{\mu }}(r) = {{{\mu }}_1} + ({{{\mu }}_2} - {{{\mu }}_1}){e^{\frac{r}{b}}}
        \end{array}$
    &   \begin{minipage}{0.3\textwidth}
        \includegraphics[width=4.5cm, height=3.2cm]{ch2_images/distribution_exponentialeasein.png}
        \end{minipage} \\
    \hline
    степенева \newline функція &
        $\begin{array}{@{}c@{}}
            {{\sigma }}(r) = {{{\sigma }}_1} + ({{{\sigma }}_2} - {{{\sigma }}_1}){r^{ - 2}} \\
            {{\mu }}(r) = {{{\mu }}_1} + ({{{\mu }}_2} - {{{\mu }}_1}){r^{ - 2}}
        \end{array}$
    &   \begin{minipage}{0.3\textwidth}
        \includegraphics[width=4.5cm, height=3.2cm]{ch2_images/distribution_quadeasein.png}
        \end{minipage} \\
    \hline
    \end{tabularx}
\end{table}


У табл.~\ref{tab:profiles_approximation} $\sigma_1$, $\mu_1$; $\sigma_2$, $\mu_2$ "--- початкові та кінцеві значення відповідних параметрів зони апроксимації; $a$, $b$, $c$, $g$, $r$ "--- параметри, що задають вигляд апроксимаційної моделі.

\section{Висновки до розділу~\ref{ch:chapter_2}}

В цьому розділі в рамках створення експрес-методу запропоновано...

\section{Список використаних джерел до розділу ~\ref{ch:chapter_2}}
% \chapter*{Список використаних джерел до розділу ~\ref{ch:review}}
% \printbibliography[title={Reference List for Section 1}]
\printbibliography[heading=none]
\end{refsection}

%        Розділ 2

\begin{refsection}

% Приклад назви розділу і мітки, на яку можна посилатися в тексті
\chapter{Алгоритмічне і програмне забезпечення для вимірювань профілів електрофізичних параметрів методом з апріорним накопиченням даних}
\label{ch:chapter_3}

\section{Програмне забезпечення для “точного” моделювання процесів вихрострумового контролю в об’єктах циліндричної форми}

Зазвичай для побудови точних електрофізичних моделей використовуються...

\section{Програмне забезпечення для створення комп’ютерних однорідних планів експериментів}\label{ch3:section2}


Для прикладу масштабування даних для реальних значень факторів взято базові значення з таблиці~\ref{tab:ch3:data_for_scaled_values} і ...
\begin{table}[htbp]
\small

\caption{Вихідні дані для створення масштабованих планів експерименту}
\label{tab:ch3:data_for_scaled_values}

\begin{center}
\begin{tabular}{|p{200pt}|c|c|c|c|}
    \hline
    & $\sigma_{51}$, См/м & $\mu_{r51}$ & $f$, Гц & $r$, мм \\ \hline
    Базове значення \newline (BaseValue) & 6990000 & 10 & 2500 & 10 \\ \hline
    Максимальне відносне відхилення & 15,00\% & 15,00\% & 20,00\% & 2,50\% \\ \hline
\end{tabular}
\end{center}
\end{table}


%----------------------------------

\begin{center}
    \small
    \begin{longtable}{|c|c|c|c|c|}
    \caption{Приклад масштабованого чотирьохфакторного плану експерименту}\label{tab:ch3:scaled_values_r4} \\
    \hline
    № & $\sigma_{51}$, См/м & $\mu_{r51}$ & $f$, Гц & $r$, мм \\ \hline
    \endfirsthead
    
    \multicolumn{5}{c}{\normalsize \bfseries Продовження таблиці \thetable} \\ [12pt] \hline
    № & $\sigma_{51}$, См/м & $\mu_{r51}$ & $f$, Гц & $r$, мм \\ \hline
    \endhead

    1 & 6,4852Е+06 & 8,7095Е+00 & 6,4852Е+06 & 8,7095Е+00 \\ \hline
    2 & 5,9705Е+06 & 1,0419Е+01 & 5,9705Е+06 & 1,0419Е+01 \\ \hline
    3 & 7,5557Е+06 & 9,1286Е+00 & 7,5557Е+06 & 9,1286Е+00 \\ \hline
    4 & 7,0410Е+06 & 1,0838Е+01 & 7,0410Е+06 & 1,0838Е+01 \\ \hline
    5 & 6,5262Е+06 & 9,5476Е+00 & 6,5262Е+06 & 9,5476Е+00 \\ \hline
    \ldots & \ldots & \ldots & \ldots & \ldots \\ \hline
    2498 & 6,3372Е+06 & 1,1383Е+01 & 6,3372Е+06 & 1,1383Е+01 \\ \hline
    2499 & 7,9224Е+06 & 1,0092Е+01 & 7,9224Е+06 & 1,0092Е+01 \\ \hline
    2500 & 7,4076Е+06 & 8,8019Е+00 & 7,4076Е+06 & 8,8019Е+00 \\ \hline
    2501 & 6,8929Е+06 & 1,0511Е+01 & 6,8929Е+06 & 1,0511Е+01 \\ \hline
    2502 & 6,3781Е+06 & 9,2210Е+00 & 6,3781Е+06 & 9,2210Е+00 \\ \hline
    \ldots & \ldots & \ldots & \ldots & \ldots \\ \hline
    4996 & 7,7743Е+06 & 9,7658Е+00 & 7,7743Е+06 & 9,7658Е+00 \\ \hline
    4997 & 7,2595Е+06 & 1,1475Е+01 & 7,2595Е+06 & 1,1475Е+01 \\ \hline
    4998 & 6,7448Е+06 & 1,0185Е+01 & 6,7448Е+06 & 1,0185Е+01 \\ \hline
    4999 & 6,2300Е+06 & 8,8944Е+00 & 6,2300Е+06 & 8,8944Е+00 \\ \hline
    5000 & 7,8153Е+06 & 1,0604Е+01 & 7,8153Е+06 & 1,0604Е+01 \\ \hline
\end{longtable} 
\end{center}


Наведений приклад показує...

\section{Висновки до розділу~\ref{ch:chapter_3}}

Алгоритмічне та програмне забезпечення для моделювання...

\section{Список використаних джерел до розділу ~\ref{ch:chapter_3}}
% \chapter*{Список використаних джерел до розділу ~\ref{ch:review}}
% \printbibliography[title={Reference List for Section 1}]
\printbibliography[heading=none]
\end{refsection}

%        Розділ 3
\begin{refsection}

% Приклад назви розділу і мітки, на яку можна посилатися в тексті
\chapter{Програмно-апаратний комплекс для вимірювання профілів електрофізичних параметрів циліндричних об’єктів контролю}
\label{ch:chapter_4}

% -------------------------------------------------------------------
\section{Апаратна частина комплексу}
% Приклад назви підрозділу

Однією з головних переваг підходу до розв’язку задачі...

\begin{figure}[htbp]
    \setlength{\unitlength}{1mm}
    \begin{center}
    \includegraphics[width=0.9\linewidth]{{ch4_images/hardware functional schematic 1.drawio.png}}
    \caption{Функціональна схема вихрострумового структуроскопу.}
    \label{fig:ch4:hardware_functional_schematic}
    \end{center}
\end{figure}


\begin{itemize}
    \item Генератор струму збудження, до якого входить генератор прямого цифрового синтезу (Direct digital synthesizer, DDS) та підсилювач сигналу сигнезатора. Він генерує синусоїдальний струм заданої частоти, який подається на обмотку збудження ВСП.

    \item ВСП з прохідним датчиком трансформаторного типу у вигляді двох котушок - котушки збудження та вимірювальної котушки.

    \item Модуль вимірювання параметрів...
\end{itemize}

% -------------------------------------------------------------------

% \begin{landscape}
    \begin{figure}[htbp]
        \setlength{\unitlength}{1mm}
        \begin{center}
        \includegraphics[width=\linewidth]{{ch4_images/sch_exitation_amp.png}}
        \caption{Електрична принципова схема підсилювача сигналу збудження.}
        \label{fig:ch4:sch_exitation_amp}
        \end{center}
    \end{figure}


    \begin{figure}[htbp]
        \setlength{\unitlength}{1mm}
        \begin{center}
        \includegraphics[width=0.9\linewidth]{{ch4_images/sch_signal_amp.png}}
        \caption{Електрична принципова схема підсилювача сигналу ВСП.}
        \label{fig:ch4:sch_signal_amp}
        \end{center}
    \end{figure}
% \end{landscape}

% landscape
\begin{landscape}
    \begin{figure}[htbp]
        \setlength{\unitlength}{1mm}
        \begin{center}
        \includegraphics[width=0.9\linewidth]{{ch4_images/sch_amplitude_and_phase_detectors.png}}
        \caption{Електричні принципові схеми фазового та амплітудного детектора сигналу ВСП з фільтрами низьких частот другого порядку.}
        \label{fig:ch4:sch_amplitude_and_phase_detectors}
        \end{center}
    \end{figure}

\end{landscape}

% -------------------------------------------------------------------

% -------------------------------------------------------------------
\section{Висновки до розділу~\ref{ch:chapter_4}}

Даний розділ присвячений опису...

\section{Список використаних джерел до розділу ~\ref{ch:chapter_4}}
% \chapter*{Список використаних джерел до розділу ~\ref{ch:review}}
% \printbibliography[title={Reference List for Section 1}]
\printbibliography[heading=none]
\end{refsection}

%        Розділ 4

\chapter*{Висновки}

В дисертаційній роботі розв'язана актуальна прикладна науково-технічна задача, яка має суттєве значення...

Головні результати дослідження, що отримано в ході розв’язання актуальної прикладної науково-технічної задачі, є наступними.

\begin{enumerate}
    \item Проведенний аналіз предметної області, а саме існуючих методів та засобів вимірювання приповерхневих профілів електрофізичних характеристик об'єктів контролю, виявлені їх недоліки та обґрунтувані перспективні нові підходи до підвищення швидкодії, ефективності та точності їх визначення. Також проведений аналіз відповідного математичного апарату  ефективного розв’язання обернених задач вимірюваного неруйнівного контролю.
   
    \item Проведено обґрунтування...
\end{enumerate}

Отже, в результаті проведених досліджень створено новий...
%       Висновки

\appendix
%% mon2017dev-app1.tex  Приклад додатку зі списком публікацій та відомостями про апробацію (для mon2017dev.tex)

\chapter{Список публікацій здобувача за темою дисертації
  та~відомості~про апробацію результатів дисертації}

% Обов'язковим додатком до дисертації є
% список публікацій здобувача за темою дисертації
% та відомості про апробацію результатів дисертації
% (зазначаються назви конференції, конгресу, симпозіуму, семінару, школи,
% місце та дата проведення, форма участі).

\repeatauthorpublications


\begin{refsection}
\nocite{
  my_halchenko2019nonlinear,
  my_halchenko2020restoration,
  my_halchenko2020construction,
  my_halchenko2020surrogate,
  my_halchenko2022measurement,
  my_trembovetska2020linear,
  my_halchenko2020methods,
  my_trembovetska2019accuracy,
  my_storchak2017eddy,
  my_storchak2018neural,
  my_storchak2018mathematical,
  my_galchenko2019surrogate,
  my_galchenko2019resource,
  my_storchak2019computational,
  my_tychkov2019problematic,
  my_storchak2019modeling,
  my_storchak2019analysis,
  my_storchak2020reconstruction,
  my_halchenko2020efficient,
  my_trembovetska2020metamodeling,
  my_tychkov2020identification,
  my_halchenko2020analysis,
  my_tychkov2020neurocomputing,
  my_trembovetska2020synthesis,
  my_storchak2020inversion,
  my_tychkov2020neuralnetwork,
  my_storchak2024systemeddy,
}

% {\Large \textbf{Custom Heading Without Section}}\par
\vspace{1em}
\centering
{ \textbf{Список публікацій здобувача за~темою~дисертації}}\par
% Add some vertical space after the custom heading
\vspace{1em}

\printbibliography[heading=none]
% \printbibliography[title={Список публікацій здобувача за~темою~дисертації}]
\end{refsection}


% TODO: Перевірити, якщо головна мова документа -- english.
\repeatapproval
%  Додаток 1 (список публікацій та відомості про апробацію)

\chapter{Акти впровадження}

\newpage
\begin{figure}[h]
    % \setlength{\unitlength}{1mm}
    \begin{center}
    \includegraphics[width=1\linewidth]{{akt_vprovadzhenia/akt_02_1.jpg}}
    \end{center}
\end{figure}

% \newpage
% \begin{figure}[h]
%     % \setlength{\unitlength}{1mm}
%     \begin{center}
%     \includegraphics[width=1\linewidth]{{akt_vprovadzhenia/akt_02_1.jpg}}
%     \end{center}
% \end{figure}
%  Додаток 2 (акти впровадження)
%\include{xampl-app2}%      Додаток 2 і т. д. ще скільки потрібно додатків

\end{document}
