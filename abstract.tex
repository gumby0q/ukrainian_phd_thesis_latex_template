%% mon2017dev-abs.tex  Приклад анотації (для mon2017dev.tex)

\begin{abstract}[
  language=ukrainian,% мова анотації
  % chapter=Реферат, % заголовок розділу або false, щоб не робити заголовок (типово Анотація/Abstract)
  % header=false     % автоматична генерація опису дисертації (типово true)
]

В дисертаційній роботі було розв’язано задачу щодо вимірювання приповерхневих радіальних профілів електрофізичних характеристик циліндричних об’єктів та описано процес створення такої системи вимірювання на основі реалізованої сурогатної моделі процесу вихрострумового контролю...


  \keywords{%
    вихрострумове вимірювання, 
    моніторинг,
    циліндричний об’єкт контролю, 
    радіальні профілі магнітної проникності і електричної провідності, 
    сурогатна модель, 
    нейромережа, 
    однорідний комп’ютерний план експерименту,
    ...
  }
\end{abstract}

\begin{abstract}[
  language=english,
  % chapter=Summary,
  % header=false
]
  

The dissertation solves the problem of measuring the subsurface radial profiles of the electrophysical properties of cylindrical objects and describes the process of creating such a measurement system based on the implemented surrogate model of the eddy current testing process...

  \keywords{%
    eddy current measurement, 
    monitoring,
    cylindrical test object, 
    radial profiles of magnetic permeability and electrical conductivity, 
    surrogate model, 
    neural network,
    uniform computer design of experiment,
    ...
  }
\end{abstract}


\begin{refsection}
\nocite{
  my_halchenko2019nonlinear,
  my_halchenko2020restoration,
  my_halchenko2020construction,
  my_halchenko2020surrogate,
  my_halchenko2022measurement,
  my_trembovetska2020linear,
  my_halchenko2020methods,
  my_trembovetska2019accuracy,
  my_storchak2017eddy,
  my_storchak2018neural,
  my_storchak2018mathematical,
  my_galchenko2019surrogate,
}


% {\Large \textbf{Custom Heading Without Section}}\par
\vspace{1em}
\centering
{ \textbf{Список публікацій здобувача за~темою~дисертації}}\par
% Add some vertical space after the custom heading
\vspace{1em}

\printbibliography[heading=none]
% \printbibliography[title={Список публікацій здобувача за~темою~дисертації}]
\end{refsection}


% \nocite{Bar98fasp1,Bar98fasp2,PrB01umc}

% \begin{bibset}% [a]
%   {Список публікацій здобувача за~темою~дисертації}
%   % {Список публікацій здобувача}
%   \bibliographystyle{gost2008}
%   %
%   % Якщо не треба нумерація з крапкою, можна закоментувати наступні три рядки.
%   \makeatletter
%   \renewcommand\@biblabel[1]{#1.}
%   \makeatother
%   \bibliography{xampl-mybib}
% \end{bibset}
